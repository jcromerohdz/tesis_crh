\chapter{Introduction} \label{introduction}


\par Interactive evolutionary computation (IEC) are systems where users replace the
fitness function in evolutionary methods \cite{eiben2015introduction}.
These systems usually apply to problems
where the fitness function is unknown or does not exist, and the result of the
optimization must be adjusted to particular needs, for example, the tastes or
desires of users. In that sense, the case study includes evolving objects
with individual characteristics, such as visual appeal as well as others where
human behavior is considered, such as optimization of teamwork
\cite{kosorukoff2002evolutionary} or creativity \cite{yu2011cooks}. In cases
where human interaction is   responsible for other aspects of the evolutionary
process, some authors classify it as evolutionary computation based on human
\cite{kosorukoff2001human} or human-based computation \cite{quinn2011human}.
Interactive evolutionary computing systems are an exciting field of research, as
they have demonstrated their ability to produce art and designs effectively
\cite{bentley1999introduction, kowaliw2012promoting, sims1991artificial,
todd1994evolutionary}.

\par However, the need for human intervention leads to several challenges for
designers of such systems, for instance, evaluations are slow, costly and
scarce, there is also the fatigue of people caused by interactions
\cite{takagi1998interactive}. The boredom that arises when users evaluate large
numbers of phenotypes (individuals) as these are very similar or just not attractive. The
performance of these systems depends on the amount of user participation, to
reach more people, some of these systems are migrating to web-based applications
depending on the help of their visitors about search with either anonymous users
or with registered users. Some of these systems implement collaborative
techniques, where some users participate in the evaluation of individuals, this
method is called collaborative-IEC \cite{secretan2008picbreeder,
seyama2016development, wagy2014collective}.

\par We believe that interactive evolutionary computing should follow a human-centered
design \cite{greenhouse2012human}, giving particular attention to volunteer
users, not only because of the explicit evaluations they perform and are
essential but because the interaction context affects the system as a whole.
Because users not only interact with the graphical interface, and with each
phenotype presented in the system, but also users interact with other users, as
well as the activities they perform for example storing a phenotype, see some
phenotype of a given collection of another known user, and so on.

\par Within these systems not all users are equal, because the human is a complex
entity \cite{mitchell2009complexity}. In this sense we could say that users\'
evaluations do not have the same weight, since this entails
that those users with a greater participation have certain influence on others,
this would produce a
type of trend in the evolutions of the phenotype and is necessary to know who
has the highest level of participation in these systems. This problem can be
addressed in different ways, but we propose to use the fuzzy logic
\cite{Zadeh1973}\cite{TakagiSugeno1983}\cite{ohsaki1998input} in order to assign
a weight to users\' preferences according to certain criteria that are within the
interaction activities carried out by users such as activities, tasks,
evaluations of phenotypes and rankings.

\par Therefore this gives us the guideline to generate a fitness assignment
strategy which weighted users according to the interaction activities mentioned
above in these systems beyond just the number of evaluations performed and the
number of views of the users. This makes users with more participations have
greater value, and therefore other users see it as a leader user about their
participations.

\par This thesis argues that a graph based user model is a viable representation of
the interactions and relations of actions that exist between users
and the phenotypes of these systems. Such relations could be evaluate, share, delete,
store and visualize entities or phenotypes. Graph-based models are currently
used in social networks to
keep track of users' interactions with media objects, places, and other users
\cite{bondy1976graph, miller2013graph, holzschuher2013performance}. Users on
social networks (e.g., Facebook) are accustomed to expressing these
complex relationships in statements such as: "Christian and Iliana are eating on
Friday's." Another example of a graph representation for this kind
of systems is the W3C activity stream 2.0 specification used to
represent activities in social web applications.

\par The knowledge generated by the graph-based user model is used to know how users
behave in this type of system with respect to their participation, that is, we
can observe the information generated by each user. In this sense, this
information is used both as part of the fitness calculation of the phenotypes in
one of the inputs in a fuzzy system and is also used as usability elements
interface through the gamification technique \cite{huotari2012defining,
deterding2011game, hickman2010total, mcgonigal2011reality}. The technique
implemented in this research is based on a rewards mechanism \cite{sutter2010browse}.
Generally rewards are granted according to the reputation that certain users
have in the system and are represented using a score system with points, levels and
leaderboards. Points or scores are rewards given to the accomplishments of
activities that need to be done by users. The levels are acquired according to
the points that are generated by users. Finally a leaderboard is where a user position
is shown with respecto to others. The above
techniques cause that users in certain way get engage in these kind of
systems, due to the competitiveness between them.

% You must state the problem here, something like.
% In order to increase user participation we propose a framwork etc.
% Im shure this is stated in other section so you just need to
% get it in here
\par The necessary intervention of users in interactive evolutionary computational
systems has inherent drawbacks arising from the very nature of the algorithms,
namely, the human fatigue caused by the interaction, and the boredom arising
when users evaluate a large number of artifacts. To tackle these issues, in this
research we propose a human-centered framework to model complex interactions on
these systems.

\par In this work a case study is presented, where a web application was
developed based on digital art where the objective of the application is to
evolve drawings. This study case has three different versions where each of them
is presented for use to different groups of users. The first version uses simple
interactive evolutionary computation techniques where users can only visualize
and evaluate phenotypes as well as access the application and have the
functionality to create collections of their creations, but there is no record
of their activities. Also the second version is based on the aspects of the
first one, but this generates information of the activities of the users such as
registration to the application, evaluate phenotypes, create collections and so
on, and the fuzzy logic paradigm is implemented.  The third version is also
based on the aspects of the two previous ones but the difference lies in how we
generate the information of interactivity with the application, since this
generates the graph-based model and the rewards mechanism is implemented, which
refers to usability aspects such as gamification, all this is implemented to be
able to study the changes of participation that exists with respect to the other
versions mentioned above. This study case is exposed in detail to the readers in
order to understand the concepts and uses of the proposed framework.



\section{Outline}

\begin{itemize}
\item \textbf{Chapter 2}, describes an in-depth study of current and related
works, presenting a general overview of interactive evolutionary computation and
their evolution in the pass and recent years, also user model approaches and
finally the gamification paradigm. This study includes Interactive evolutionary
computations methods and techniques to understand how they work, as well as
common problems found in these systems. On the other hand this study also includes user
modeling methods in order to understand how to create and apply them for certain
applications. Finally we discussed a gammification paradigm to
understand how it work and subsequently apply this for the necessities of this
work.

\item \textbf{Chapter 3}, describes the fundamental
concepts that form the basis of the proposed human-centered framework.

% Why are you hard coding the number of chapters? You should use \ref{} for this.
\item \textbf{Chapter 4}, presents a human-centered framework for interactivity
Web-based applications, this proposed framework involves different paradigms in
order to increase user participation. This chapter also includes the overall
explanation of how the functionality the proposed framework works.

\item \textbf{Chapter 5}, the case study is presented along with the
explanation of the experimentation for this work. The experiments were realized
using several versions of the same application in order to see which of them
have most user participation.

\item \textbf{Chapter 6}, The results of the case study are presented. This
chapter includes all the data obtained from the experiments, these data are
presented to see which of the versions meets the best performance regarding the
increase of participation of users.

\item \textbf{Chapter 7}, finally the conclusions and future work of this
research are exposed.
\end{itemize}

%%% Correct this following paragraph.
% At the end of this  includes appendices that describe
% detailed technical aspects about installation EvoDrawings application on heruku
  %\textit{(appendix \ref{appendixa})},
% The rules of the fuzzy inference system are presented in  %\textit{(appendix \ref{appendixb})},
%and experiment study materials used to obtain the test results are in  %\textit{(appendix \ref{apendixc})}.
