\chapter{Introduction} \label{introduction} 


\par It is a reality that the World Wide Web in recent years, is growing
exponentially, which means the presence of millions of users on Web sites, Web
applications, Web systems, etc. []. There is a wide variety of Web systems,
where we have different users interacting with them. These users have different
goals when using these Web systems. For example to serching  in Google
\cite{google} for a particular topic, make a reservation for a room in a luxury
resort, check their bank account or simply checking their Facebook account
status [].This variation of users represents a complex diversity as individuals
[]. This diversity lies in the fact that users have different skills, interests,
preferences, knowlage defferent  ways of thinking and learning [].For this
reason users most to interact with the information presented by existing Web
systems. \par When we intend to customize any element in Web system,  it is
necessary to some personal information about the user. This information is a
collection of needs, characteristics, tastes, preferences among others.  This
information allows designers to build a representation of knowledge about the
users. This is what is known as user modeling (UM).    
\par  In order to
understand specific users a user model must be constructed, this can be simple
as a profile where the basic knowledge recorded. Or it can be as complex as
complete representation of it's characteristics, needs, interests and
preferences. In order to understand specific users. The main goal of user
modeling is to represent aspects of the real world of the user's in an autonomous
way.

\par On the other hand, interactive evolutionary computation (IEC)
is a branch of evolutionary computation where users become a part of the
evolutionary process by replacing the fitness function; evaluating individuals
of a population based on their personal preferences[13]. These evaluations are
subjective according to the user point of view based on their perceptions,
interests and desires.  
\par Normally such systems require users to evaluate large amounts of individuals 
actively, causing them to lose interest in participating by the fatigue that 
is generated[13]. Nowadays some of these systems
are migrating to  Web technologies looking for volunteer users to collaborate
in the evaluation for distribute the load and lower the fatigue. Having Web-
based interactive evolutionary systems opens the possibility of linking to social networks 
platforms in order to involve a larger number of users to assist in
the evaluation of individuals produced by these systems.

\par This research presents a multidisciplanary method that involves several techniques 
such as graph-based user modeling, fuzzy logic and human-interaction in the context 
of a Web-based interactive evolutionary computation. The purpose of this research 
is to increase voluntary participation of the user's  using the proposed method.

This thesis is organized as follows:
\begin{itemize}
\item  \textbf{Chapter 2: State of the art.}{The related work is presented.}
\item  \textbf{Chapter 3: Background.}{The theory and background are preseted.} 
\item  \textbf{Chapter 4: Proposed Method.}{The proposed of the method is described, its characteristics and main objectives. } 
\item  \textbf{Chapter 5: Study Case.}{Description of the scenarios of the experiments that were used in this research.}
\item  \textbf{Chapter 6: Results.}{The experimental results are shown.}
\item  \textbf{Chapter 7: Conclusions and Future Work.}{Conclusions and future work are presented.}
\end{itemize}