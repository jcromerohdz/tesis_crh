\chapter{Introduction} \label{introduction} 


\par It is a reality that the World Wide Web in recent years, is growing
exponentially, which means the presence of millions of users on Web sites, Web
applications, Web systems, etc. []. There is a wide variety of Web systems,
where we have different users interacting with them. These users have different
goals when using these Web systems. For example do a search in Google \cite{google} of
particular topic, make a reservation for a room in a luxury resort, check your
bank account or simply view your status on your Facebook account [].This
variation of users represents a complex diversity as individuals []. This
diversity lies in different skills, interests, preferences and ways of thinking,
learning and knowledge [].For this reason users need different ways to interact
with the information presented by the great variety of Web systems that exist.
\par When we intend to customize any element in Web system, we need to
know the user's personal information. This information is a collection of needs,
characteristics, feelings, tastes, etc. This information is required to be able
to form the representation of knowledge about users. This is what is known as
user modeling (UM).  
\par A user modeling can be as simple as a profile
systems where is basic knowledge of users. Also  can be as complex as represent
its characteristics, needs, interests, ways to feel. In order to understand
specific users. The main goal of user modeling is to represent aspects of the
real world of the user's in  autonomous automatically way.  

\par Interactive evolutionary computation (IEC)
is a branch of evolutionary computation where users become a part of the
evolutionary process by replacing the fitness function; evaluating individuals
of a population based on their personal preferences[13]. These evaluations are
subjective according to the user point of view based on their perceptions,
interests and desires.  
\par Normally such systems require users to evaluate large amounts of individuals 
iteratively, causing them to lose interest for participate by fatigue that 
is generated[13]. Nowadays some of these systems
are migrating to  Web technologies looking for volunteers users to collaborate
in the evaluations for distribute the load and lower the fatigue. Having Web-
based interactive evolutionary systems open the possibility for linked to social
platforms in order to involve the largest number possible of users to assist in
the evaluation of individuals produced by these systems applications.

\par In this research we present a method where we use a combination of techniques 
such as user modeling, fuzzy logic and usability (gamification) in the context 
of Web-based interactive evolutionary computation. The purpose of this research 
is to increase voluntary participation of the user's  using the proposed method.

This thesis is organized as follows:
\begin{itemize}
\item  \textbf{Chapter 2: State of the art.}{The related work is presented.}
\item  \textbf{Chapter 3: Background.}{The theory and background are preseted.} 
\item  \textbf{Chapter 4: Proposed Method.}{The proposed of the method is described, its characteristics and main objectives. } 
\item  \textbf{Chapter 5: Study Case.}{Description of the scenarios of the experiments that were used in this research.}
\item  \textbf{Chapter 6: Results.}{The experimental results are shown.}
\item  \textbf{Chapter 7: Conclusions and Future Work.}{Conclusions and future work are presented.}
\end{itemize}