\chapter{Introduction} \label{introduction}


\par It is a reality that the World Wide Web in recent years, is growing
exponentially, which means the presence of millions of users on Web sites, Web
applications, Web systems, etc. \cite{albert1999internet}. There is a wide
variety of Web systems, where we have different users interacting with them.
These users have different goals when using these Web systems. For example to
serching  in Google \cite{google} for a particular topic, make a reservation for
a room in a luxury resort, check their bank account or simply checking their
Facebook account status \cite{cockburn2001web}.This variation of users
represents a complex diversity as individuals \cite{zhao2006internet}. This
diversity lies in the fact that users have different skills, interests,
preferences, knowlage defferent  ways of thinking and learning. For this reason
users most to interact with the information presented by existing Web systems.
\par When we intend to customize any element in Web system,  it is necessary to
some personal information about the user. This information is a collection of
needs, characteristics, tastes, preferences among others.  This information
allows designers to build a representation of knowledge about the users. This is
what is known as user modeling (UM) \cite{razmerita2009user} \cite{fischer2001user}.

\par  In order to understand specific users a user model must be constructed,
this can be simple as a profile where the basic knowledge recorded. Or it can be
as complex as complete representation of it's characteristics, needs, interests
and preferences. In order to understand specific users. The main goal of user
modeling is to represent aspects of the real world of the user's in an
autonomous way.

\par On the other hand, interactive evolutionary computation (IEC)
is a branch of evolutionary computation where users become a part of the
evolutionary process by replacing the fitness function; evaluating individuals
of a population based on their personal preferences
\cite{takagi2001interactive}. These evaluations are subjective according to the
user point of view based on their perceptions, interests and desires.

\par Normally such systems require users to evaluate large amounts of
individuals actively, causing them to lose interest in participating by the
fatigue that is generated \cite{takagi2001interactive}. Nowadays some of these
systems are migrating to Web technologies looking for volunteer users to
collaborate in the evaluation for distribute the load and lower the fatigue.
Having Web-based interactive evolutionary systems opens the possibility of
linking to social networks platforms in order to involve a larger number of
users to assist in the evaluation of individuals produced by these systems.

\par In this research we discuss that a graph model is a viable representation for
relationships and interactions such as user and phenotype knows as individual in
evoluationary algorithms, and also for user to user. The model can used, within
the evolutionary algorithm, action performed by every user, which is tag, rate,
share or event delete the individual (phenotype). In this sense, this actions
can be used to assign fitness to a given solution.

\par Graph based models are currently used in social networks to keep track of
user interaction with media objects, places, and other users. Users of social
networks (for instance the Facebook graph) are accostumed to express these
complex relationship in sentences such as: "Christopher and Samantha are eting
lunch at Fryday's". Other example is the W3C activiy stream 2.0 specification
used for representing activities common in social web aplications
\cite{snell2014json}.

\par This research presents a user-cetered framework that involves several
techniques such as graph-based user modeling, fuzzy logic and human-interaction
in the context of a Web-based interactive evolutionary computation. The purpose
of this research is to increase voluntary participation of the users  using the
proposed framework.

\section{Outline}

\begin{itemize}
\item \textbf{Chapter 2}, describes an in-depth study of current and related
works, presenting a general overview of Interactive evolutionary computation and
their evolution in the pass and recent years, also user model approaches and
finally gamification paradigm. This study includes Interactive evolutionary
computations methods and techniques to understand how they work, as well as
their problems of these systems. On the other hand this study also includes user
modeling methods in order to understand how to create and apply them for certain
necessities. Finally we discussed a gammification usability paradigm to
understand how it work and subsequently apply this for the necessities of this
work.

\item \textbf{Chapter 3}, describes the fundamental
concepts that form the basis of the proposed user-cetered framework.

\item \textbf{Chapter 4}, presents a user-centered framework for interactivity
Web-based applications, this proposed framework involves different paradigms in
order to increase user participation. This chapter also includes the overall
explanation of how the functionality the proposed framework works.

\item \textbf{Chapter 5}, the case study is presented along with the
explanation of the experimentation for this work. The experiments were realized
using several versions of the same application in order to see which of them
have most user participation.

\item \textbf{Chapter 6}, The results of the case study are presented. This
chapter includes all the data obtained from the experiments, these data are
presented to see which of the versions meets the best performance regarding the
increase of participation of users.

\item \textbf{Chapter 7}, finally the conclusions and future work of this
research are exposed.
\end{itemize}

At the end, this thesis includes appendices that describe
detailed technical aspects about installation EvoDrawings application on heruku
  %\textit{(appendix \ref{appendixa})},
The rules of the fuzzy inference system are presented in  %\textit{(appendix \ref{appendixb})},
%and experiment study materials used to obtain the test results are in  %\textit{(appendix \ref{apendixc})}.
