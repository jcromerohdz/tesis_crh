%% Los cap'itulos inician con \chapter{T'itulo}, estos aparecen numerados y
%% se incluyen en el 'indice general.
%%
%% Recuerda que aqu'i ya puedes escribir acentos como: 'a, 'e, 'i, etc.
%% La letra n con tilde es: 'n.

%
%MG: En el resumen debes hablar de todo el trabajo revisa:
%% http://www.slideshare.net/jjmerelo/cmo-escribir-y-publicar-trabajos-cientficos-17948694

% Da una breve introducción al tema (problema)
% Los métodos que pleanteamos aquí son {metodología}
% para [mejorar, evitar, comprobar] {problema}
% Se probó de la manera {x} que los objetivos se alcanzaron
% con los resultados siguientes {x}

% Al integrar estas técnicas el desempeño mejoró en {x} respecto a {y}
% de manera estadísticamente significativa logrando para los datos de prueba
% un error absoluto de XX.
% Es necesario comentar los resultados con el error?..




\prefacesection{Dedicatoria}
Dedico este trabajo a mi esposa Iliana Dur\'on Landero por el apoyo incondicional
que brindaste durante todo el proceso.\\


---Sabemos muy poco, y sin embargo es
sorprendente que sepamos tanto, y es todav\'ia mas sorprendente que tan poco
conocimiento nos de tanto poder.
\qauthor{Bertrand Russell}
---Despu\'es de todo, cualquier tipo de conocimiento implica auto-conocimiento.
\qauthor{Bruce Lee}
---El conocimiento descansa no solo sobre la verdad sino tambi\'en sobre el error.
\qauthor{Carl Jung}
---El conocimiento depende del tiempo, mientras que el saber no. El
conocimiento es una fuente de acumulaci\'on, de conclusión, mientras que el saber
es un continuo movimiento.
\qauthor{Bruce Lee}


% Dedico este trabajo a mi esposa Iliana Dur\'on Landero por el apoyo incondicional
% que brindaste durante todo el proceso.\\
% ``Sabemos muy poco, y sin embargo es
% sorprendente que sepamos tanto, y es todav\'ia mas sorprendente que tan poco
% conocimiento nos de tanto poder.'' -- Bertrand Russell\\
% ``Despu\'es de todo,cualquier tipo de conocimiento implica auto-conocimiento''-- Bruce Lee \\
% ``El conocimiento descansa no solo sobre la verdad sino tambi\'en sobre el error.''-- Carl
% Jung \\
% ``El conocimiento depende del tiempo, mientras que el saber no. El
% conocimiento es una fuente de acumulación, de conclusión, mientras que el saber
% es un continuo movimiento.''-- Bruce Lee
