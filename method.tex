\chapter{Proposed method}

This chapter we explain the proposal method describing all the differents
techniques and  tools to apply user modeling in interactive evolutionary
computation area using fuzzy logic, graphs theory and "gamification" paradigm.

A fundamental aproach of this research is to develop a user modeling for
interactive evolutionary computation in order to increase the user participation
and to minimize the amount of evaluations for the evolutionary process in given 
web-based application.

The web-based IEC applications in this work are using
“EvoSpace framework” mentioned in chapter X. It is important to mention that the
model it is limited to the web-based interactive evolutionary computation context.

In following sections we explain the techniques mentioned above.


\section{Graph-based user model} 

To develop a user model, it was necessary to know the factors that are involved in
interactive evolutionary computation system . The key factors
that drive modeling users are described here:

Individuals: The individuals are agents ( entities ) that create the population
of the interactive algorithm.

Users: The users are the entities ( agents )
involved in the evaluation of individuals generated in the interactive genetic
algorithm replacing the fitness function. 
User interface: It is the platform where users make their evaluations in an 
easy and fast way. 
User activity: They are the set of actions that the user performs in the application.

The following user modeling based on graph theory was conceived knowing these factors.

We can find the vertices that are defined by the users, individuals, and
collections as we can observe in figure 1. The users has the following
properties: The following user modeling based on graph theory was conceived
knowing these factors.  Identifier Creation Date Description

The  individuals has properties as well: Identifier Creation Date Name Type

The collections has the following properties: Identifier Creation Date Name Type


The edges in Figure 1 represents the relationship between the vertices.

The edge “KNOWS” has the following properties: Identifier Creation Date

The edge “HAS” has the following properties: Identifier Creation Date

The edge “LIKES” has the following properties: Identifier Creation Date Rating

Finally the edge “PARENT” has the following properties: Identifier Creation Date

Imagen


When users interact with the application, a graph theory model is formed. This
model is described below.

The model starts when users access the application for the first time through
his "Facebook" account; after completion of this action a node, with the
identifier that gives the social platform knowledge, is created. Internally, the
call is made to the node object with the creation method where the identifier
parameters, creation date, type, and name are executed.  If users had previously
logged, then the application searches through the identifier to prevent the node
replication.

Later when users start the participation with the individual's evaluation, in
the application an edge is created; this edge is called "LIKES" and represents
the relationship between the individual node and the node of the user who is
evaluating.

It is important to mention that an EvoSpace initial population of individuals
replication, in the form of graphs, is made. This replication has the main goal
to generate the individual's nodes and perform the current assessment.

If the user creates a collection in the application, a collection node type,
with attributes identifier and name, is generated. Now if the user intends to
store a digital painting according to his preference, in one of the collections
previously created, a relationship called “HAS” is created; this relationship is
between the collection node and the individual node which represents the paint
chosen to be stored.

The relationship between the user node and nodes that represent all his friends
is represented by an edge called "KNOWS." To create this edge, it is necessary
to search for users who have installed the application "Facebook" and
subsequently create this relationship.

To create the edge that represents the "KNOWS" relationship is necessary to
search for users who have installed the application "Facebook" and subsequently
create this connection between the user node and nodes that represent all the
friends.

\section{Interface}
In particular, we shall explain how it was necessary to create an application called “EvoDrawings” to improve the way users interact “collaborate” with an interactive evolutionary computing system. One important class of this application is the user interface, which is the model we will explain in this chapter.

EvoDrawings is a web-based application of interactive evolutionary computing system. In these days when social media platforms has become the perfect tool for masses to exchange information/ideas, interests and creations, EvoDrawings include as a common requirement to the users to have an account of the social platform “Facebook” to have access to the application and consequently participate in it.

The main point to have users logged in the application is that they can evaluate individuals that are formed as a digital paint composed for a single chromosome of nine positions of real numbers.
Each post of the chromosome represents some figure, color, performing behavior, etc. In figure X we can observe a chromosome composed of 15 elements from a digital paint. Once the individual is shown, the users can proceed to evaluate the individual subjectively. This means that according to the user's preferences the user can print his taste in all the evaluations.
The evaluation method consists of giving the desired rating through an interactive visual component of five stars. The interactive visual component allows the user to select from one to five stars to evaluate the individual, having the one star as a slightly liking rate and the five stars like a total satisfaction rate. 
The application enables the user to visualize his social network and the personal evaluation of the users from his social net. In addition to this, the users can create collections with the main goal to allow the users to stock the more pleasing digital paints based on his preferences. 
Also, the user interface contains a section called “About” where the application explains to users in a general way what the application is all about.



We can find the vertices that are defined by the users, individuals, and collections as we can observe in figure 1. The users has the following properties:

\section{User activity}
The user activity is not so different from how the graph is formed, with the difference that the information generated is formed using the standard specified by "JSON activity stream 1.0," which is stored in the engine database NoSQL "Redis ."

An activity consists of 4 elements: an actor, a verb, an object and a target. The activity generates the history of a user and performs an action on an object. For example - "Christian likes the individual 3" or "Mario created a collection". In most cases, the components will be explicit, but may also be implicit.

The primary goal of this specification is to provide sufficient metadata about the activity, to help the consumer of these data to present the information to the user, in a straightforward and user-friendly format. This involves building simple sentences on the activity that is happening, as well as the visual representation of the activity.

So far, we can say that an "Activity Stream" is a collection of one or more activities of an individual (user).
This specification does not define the relations between the activity within the collection, therefore remains to the interpretation of the user who implements it.

\section{Database for interactive evolutionary computation}
The database for interactive evolutionary computation consists of two databases, as we observe in figure X. In one hand we have the EvoSpace database that uses Redis database engine that is already explained in Chapter X section X.

This database contains the structure of individual user data, where every user participation is being stored as well as the fitness of the individual, the user identifier, the representation of the chromosome, etc. 

It is necessary to have this information because it is used for fuzzy inference block, which is explained in another chapter. On the other hand, we have a relational database, where basic user information is stored, such the ID, the email, the session, etc. This database also stores everything needed to meet the requirements of login for "Facebook". One important additional information is that contains the structure for storing collections as we can see in Figure X.

\section{Fuzzy Inference}
In this chapter, we will focus on the explanation of fuzzy inference block. This block uses fuzzy inference to acquire a parameter having the weight function to adjust the participation in interactive evolutionary computing applications. This parameter is used in the decision block which will be explained later. Additionally, the parameter is also acquired from the information generated in different databases.

The fuzzy inference system is Mamdani type and is composed of three inputs and one output. Where entries are defined by the preference variable that is also composed of three functions of triangular membership; these features are called low, medium, high, and have ranged from 1 to 5. This range is given by the preference that the user assigns to the individual at the time of assessment.

We also consider the input variable called "experience" and it is defined by three functions of triangular membership under the name of low, medium and high in a range of 1 to 100. The range of this variable are acquired from the activities that the user performs in the application, and empirically with each activity, a score is assigned. For example, if the user makes a login, a three-point score is assigned, as well if the user evaluates an individual, a two-point score is assigned, all the activities has a score punctuation and also the user has a score limit of one hundred points.

The third variable which is called “ranking” it is also defined by three triangular membership functions with the name of low, medium, high, in a range of 1 to 30. This range is defined by a ranking process as well as is also performed in video games; the range is adjusted according to the user participation.

This involvement is acquired from all the cardinality of the graph that the user has and passes through the logarithmic equation X that calculates the value of ranking. Finally, the exit "fuzzyrate" is in a range of 1 to 100 defined by three triangular membership functions with the name of bad, normal and good. In figure X we can see this fuzzy inference system.

Falta imagen aqui

We need the rules here please. ??????


\section{Decision Making}
The decision block it is defined by equation 1 representing the value of fitness that takes the individual to be evaluated by the user, as well as everything else that makes in the application.

equation please

Where  represents the fuzzy fitness function for all users who have evaluated an individual of the population in particular.
Likewise  represents the range that a user assigned to the individual according to their preferences. The function  represents our fuzzy inference and is defined by the equation 2.

equation please

Where  is still the range that users assigned to a particular individual. The variable represents the user experience, which is a function that is defined in equation 3

equation please

Where  represents user activity. The variable  represents the taste generated of the verb "like" from the user activity stream.
The variable  represents the access that like the variable  is obtained from the verb "join" from the user activity stream. The variables are represented by the verbs "save" and "open" from the user activity stream.

equation please

In equation 2, we can find the variable  as the last entry and represents a range of the user; this variable is defined by a logarithmic scale where intervenes our user modeling based on graphs and is defined by Equation 4.

equation please

In equation 4, the variable  represents the scale, levels represents the highest level that the users can have. The variable  represents the endpoints, which are the maximum points that a particular user can have.

In order to acquire the user's level of range, a function floor is calculated, the main point of this is to increase the difficulty for the high ranked users to level up. In other words the expert users needs to have more participation if they want to raise their level range. This function is defined by equation 5

equation please

Where  represents the level of range,  represents the scale and  are all the participations  that the user has made. These participations represents the degree that the user has within his own graph and is defined by the vicinity of its vertex  that is given by the adjacent vertices to , defined in Equation 6.

equation please

In this case the degree of the vertex is number of neighbors of: 

equation please

we represent this participation as we can see in figure x.









