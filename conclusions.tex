\chapter{Conclusions and future work} \label{sec:5}

\section{Conclusions}

Using a user model in Web-based interactive evolutionary computation  overall
with the different approaches such as fuzzy logic and gamification it
demonstrated  in experiment ED03 that the users increase their
participation with respect to other versions (ED01, ED02).

In this sense the results have shown a phenomenon in the users which is
competitiveness. This phenomenon occurs naturally because as human beings is our
nature to be competitive regardless of the topic or activity that we assign.
This gave support to users return to evaluate more individuals
within the experiment ED03 and consequently the participation increase
exponentially.

In this research work also found that the way individuals was presented to be
evaluated and how to evaluate them helped the user to take their evaluations so
easy and quick. This means that users evaluated on average 40 or more
individuals in an iteration, reducing the risk of demotivation assessment and
therefore lose interest in participation. However, it was concluded that the
biggest problem of interactive evolutionary computing systems remains on user
fatigue.

The fatigue can be generated by many factors, such as how to evaluate
individuals, the subject of the application, how to present the individuals, the
objective within the application, expertise by users, and more. The resulting
method of this research helps motivate users on the issue of participation in
interactive evolutionary computing applications.


\section{Future work}

Competitiveness is among users regardless of the level of expertise they have.
Still psychological test would be necessary to see the level of fatigue that
users acquire in systems of this context. These psychological tests are beyond
the scope of this thesis since the main objective of the research was about
measuring the participation of users through a graph-based user modeling, which
finally answers  the hypothesis of this research. However this research allows
adaptation and implementation of other techniques and research that enrich this
work with multidisciplinary teams to determine that this method and other
methods can reduce the user's fatigue in the context of Web-based interactive
evolutionary systems.

Another way to attract and increase user’s participations is to work with
techniques of intelligent interfaces and natural interfaces in order to the
user's evaluate more naturally, thus creating more natural and intelligent way
to evaluate individuals, with which the user may feel more comfortable in
evaluating individuals. For instance  gesture-based interfaces, visual
interfaces using sensors (cameras, Kinnect) to detect the time or emotion felt
by the user when presented with an individual who needs to be evaluated.

The combination of these techniques generates more robust methods in
competitiveness among users using this type of  interfaces, for example
implement rewards medals type or representative plates when the user reaches a
certain level of participation it increases the interest to continue
participating and improving the level of expertise within the systems, as well
as to share their achievements in their social networks. In this sense we
believed that no matter the topic of interactive evolutionary computation
systems users will participate collaboratively having fun without knowing in
depth is participating.
