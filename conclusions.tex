\chapter{Conclusions and future work} \label{sec:5}

\section{Conclusions}

Using a user model in Web-based interactive evolutionary computation  overall with the different approaches such as fuzzy logic and gamification it demonstrated  in experiment EvoDrawing03 that the users increase their participation with respect to other versions (EvoDrawing01, EvoDrawin02).

In this sense the results have shown a phenomenon in the users which is competitiveness. This phenomenon occurs naturally because as human beings is our nature to be competitive regardless of the topic or activity that we assign [Reference]. This gave support to users return to evaluate more individuals within the experiment EvoDrawings03 and consequently the participation increase exponentially. 

In this research work also found that the way individuals was presented to be evaluated and how to evaluate them helped the user to take their evaluations so easy and quick. This means that users evaluated on average 40 or more individuals in an iteration, reducing the risk of demotivation assessment and therefore lose interest in participation. However, it was concluded that the biggest problem of interactive evolutionary computing systems remains on user fatigue.

The fatigue can be generated by many factors, such as how to evaluate individuals, the subject of the application, how to present the individuals, the objective within the application, expertise by users, and more. The resulting method of this research helps motivate users on the issue of participation in interactive evolutionary computing applications.


\section{Future work}

