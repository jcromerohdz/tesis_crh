\chapter{Conclusions and future work} \label{sec:5}

In this chapter, the conclusions of this research are presented as well some of
the possible future works.

\section{Conclusions}

Collaborative interactive evolutionary computation (C-IEC)
% Volunteer-based was not covered that much in the thesis (as I recall), may be only talk
% about C_IEC
web-based systems involve the dynamic interaction of many entities and
artifacts. Employing a human-centered approach will allow researchers to
understand and visualize this kind of systems better. In this research a
human-centered software framework was proposed and validated through the
implementation and refinement of a C-IEC web-based application.
The framework proposed in this thesis enabled the implementation of
a gamification technique to improve engagement in
a case study which provides a common arena where users are aware of the
activities of other users in the social neighborhood.

In concordance with the results obtained in other browser-based volunteer
systems, after applying the gamification techniques, user participation
increased. Also user participation followed the same pattern as previous systems
and it was fitted to a Weibull distribution.

In this thesis, we also found that the way individuals are presented to be
evaluated and the intarfeace to do, helped users in making those evaluations.
% There was also an usability study
Users evaluated on average 40 or more individuals in an iteration,
reducing the risk of demotivation assessment and
therefore lose interest in participation. However, it was concluded that the
biggest problem of interactive evolutionary computing systems remains on user
fatigue.

% Put here also a list of papers that resulted from your work including the CEC paper
% and GECCO, additional to the Book Chapters
From the result of this research work the following articles were published:
\begin{enumerate} \item \textit{User Modeling for Interactive Evolutionary
Computation Applications Using Fuzzy Logic. JC Romero, Mario Garc\'ia-Vald\'ez.
Recent Advances on Hybrid Intelligent Systems. Springer Berlin Heidelberg.
(2013)} \item \textit{ EvoSpace-i: a framework for interactive evolutionary
algorithms. Mario Garc\'ia-Vald\'ez, 	Juan J. Merelo, 	Leonardo Trujillo,
Francisco Fernández-de-Vega Jos\'e C. Romero, 	Alejandra Mancilla. GECCO '13
Companion Proceedings of the 15th annual conference  companion on Genetic and
evolutionary computation. Amsterdam, The Netherlands. (2013).} \item
\textit{Using a Graph Based Database to Support Collaborative Interactive
Evolutionary Systems.  JC Romero, Mario Garc\'ia-Vald\'ez. Recent Advances on
Hybrid Approaches for Designing Intelligent Systems. Springer International
Publishing Switzerland. Volume 547 of the series Studies in Computational
Intelligence pp 581-591. Switzerland. (2014).} \item \textit{A Fitness
Estimation Strategy for Web Based Interactive Evolutionary Applications
Considering User Preferences and Activities Using Fuzzy Logic. J.C. Romero,
Mario Garc\'ia-Vald\'ez. Design of Intelligent Systems Based on Fuzzy Logic,
Neural Networks and Nature-Inspired Optimization. Springer International
Publishing Switzerland. Volume 601 of the series Studies in Computational
Intelligence pp 507-516. (2015).}
\item \textit{Modeling Human Interactions in
Collaborative Interactive Evolutionary Computation.  Mario Garc\'ia-Vald\'ez,
Christian Romero, Juan. J. Merelo, Alejandra Manzilla.  In Proceedings of the Genetic and Evolutionary Computation
Conference 2017, Berlin, Germany, July 15–19, 2017 (GECCO ’17).}
\item \textit{Exploiting the Social Graph: Increasing Engagement in a Collaborative Interactive Evolution Applicati.
Mario Garc\'ia-Vald\'ez,
Christian Romero, Juan. J. Merelo, Alejandra Manzilla.  In Proceedings of IEEE Congress on Evolutionary Computation 2017
Donostia - San Sebastián, Spain
June 5-8, 2017.}

\end{enumerate}

\section{Future work}

One of the interesting future lines of work would be to look a bit more closely at the
behavior of users as they are rating artifacts in the web system. These initial
experiments hint at a possible power law, which might indicate that the IEC
system could be selforganizing, a process that would allow it to reach a
critical state, as has been found in software repositories, for instance \cite{merelo2016self}.
The dynamics of this kind of system are fundamentally different, and our future
research will include exploring these aspects of the system.

Another line of work would be to study the possible negative effects of using gamification
techniques to improve engagement, like cheating or literally gaming the system
to defeat competition. We already found some hints of this behavior at the
beginning of the release of this system, but more subtle effect could be taking
place. Finally, the refinement of the proposed Human-Centered framework will
need more case studies and further multidisciplinary research.

Another way to attract and increase user' s participations is to work with
techniques of intelligent interfaces and natural interfaces in order to the
user's evaluate more naturally, thus creating more natural and intelligent way
to evaluate individuals, with which the user may feel more comfortable in
evaluating individuals. For instance  gesture-based interfaces, visual
interfaces using sensors (cameras, Kinnect) to detect the time or emotion felt
by the user when presented with an individual who needs to be evaluated.

Competitiveness is among users regardless of the level of expertise they have.
Still psychological test would be necessary to see the level of fatigue that
users acquire in systems of this context. These psychological tests are beyond
the scope of this thesis since the main objective of the research was about
measuring the participation of users through a proposed method. However, this
research allows adaptation and implementation of other techniques and research
that enrich this work with multidisciplinary teams to determine that this method
and other methods can reduce the user's fatigue in the context.

% Maybe exploting the Social Graph in other ways, for instance to recommend new
% individuals to Users etc.

Also a future work could be to exploit the social graph, such as following users
with some degree of influence in the social network in order to recommend
phenotypes and somehow see if there is some kind of trend in the evolution of
phenotypes.
