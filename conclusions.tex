\chapter{Conclusions and future work} \label{sec:5}

\section{Conclusions}

Volunteer-based and collaborative interactive evolutionary computation (C-IEC)
web-based systems involve the dynamic interaction of many entities and
artifacts. Employing a human-centered approach will allow researchers to
understand and visualize this kind of systems better. In this research a
human-centered software framework was proposed and validated through the
implementation and refinement of a C-IEC web-based application. This framework
enabled the implementation of a gamification technique to improve engagement in
a case study which provides a common arena where users are aware of the
activities of other users in the social neighborhood.

In concordance with the results obtained in other browser-based volunteer
systems, after applying the gamification techniques, user participation
increase. Also the users follows the same pattern and it was fitted to a Weibull
distribution.

In this research work also found that the way individuals was presented to be
evaluated and how to evaluate them helped the user to take their evaluations so
easy and quick. This means that users evaluated on average 40 or more
individuals in an iteration, reducing the risk of demotivation assessment and
therefore lose interest in participation. However, it was concluded that the
biggest problem of interactive evolutionary computing systems remains on user
fatigue.


\section{Future work}

One of the interesting future lines of work would be to look a bit more closely at the
behavior of users as they are rating artifacts in the web system. These initial
experiments hint at a possible power law, which might indicate that the IEC
system could be selforganizing, a process that would allow it to reach a
critical state, as has been found in software repositories, for instance \cite{merelo2016self}.
The dynamics of this kind of system are fundamentally different, and our future
research will include exploring these aspects of the system.

Another line of work would be to study the possible negative effects of using gamification
techniques to improve engagement, like cheating or literally gaming the system
to defeat competition. We already found some hints of this behavior at the
beginning of the release of this system, but more subtle effect could be taking
place. Finally, the refinement of the proposed Human-Centered framework will
need more case studies and further multidisciplinary research.

Another way to attract and increase user’s participations is to work with
techniques of intelligent interfaces and natural interfaces in order to the
user's evaluate more naturally, thus creating more natural and intelligent way
to evaluate individuals, with which the user may feel more comfortable in
evaluating individuals. For instance  gesture-based interfaces, visual
interfaces using sensors (cameras, Kinnect) to detect the time or emotion felt
by the user when presented with an individual who needs to be evaluated.

Competitiveness is among users regardless of the level of expertise they have.
Still psychological test would be necessary to see the level of fatigue that
users acquire in systems of this context. These psychological tests are beyond
the scope of this thesis since the main objective of the research was about
measuring the participation of users through a proposed method. However, this
research allows adaptation and implementation of other techniques and research
that enrich this work with multidisciplinary teams to determine that this method
and other methods can reduce the user's fatigue in the context.
