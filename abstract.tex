

\prefacesection{Resumen}

La intervención necesaria de los seres humanos en sistemas de computaci\'on
evolutiva interactiva tiene inconvenientes inherentes que surgen de la
naturaleza misma de los algoritmos, como por ejemplo la fatiga causada por la
interacci\'on y el  aburrimiento que surge cuando los usuarios eval\'uan un gran
n\'umero de artefactos. Para abordar estos problemas, en esta trabajo de
investigaci\'on proponemos un marco de trabajo centrado en el ser humano para
modelar la interacciones complejas de estos sistemas. Se presenta un caso de
estudio donde el  modelo se aplica en el desarrollo de un sistema evolutivo
interactivo colaborativo.  Ambos detalles de implementaci\'on como conceptuales
son expuestos para medir e incrementar la participaci\'on de los usuarios
participantes. Nuestros experimentos demuestran que el modelo puede aplicar con
éxito la  t\'ecnica de competencia basadas en videojuego(gamification) para
desarrollar el aumento de atenci\'on de los usuarios, la cual implica que esta
t\'ecnica pude ser utilizada para disminuir la fatiga y el aburrimiento de los
usuarios, y as\'i aumentar el rendimiento del sistema evolutivo interactive.




\prefacesection{Abstract}

The necessary intervention of humans in interactive evolutionary computational
systems has inherent drawbacks arising from the very nature of the algorithms,
namely, the human fatigue caused by the interaction, and the boredom arising
when users evaluate a large number of artifacts. To tackle these issues, in this
paper we propose a human-centered framework to model complex interactions on
these systems. A case study is presented where the model is applied in the
development of a collaborative evolutionary interactive system. Both conceptual
and implementation details are provided where the technique is used to measure
and increase user engagement and participation. Our experiments show that the
model can be successfully applied in a gamification technique developed to
increase user engagement, which implies that this technique can successfully be
used to decrease user fatigue and boredom, and thus increase the performance of
the interactive system.
