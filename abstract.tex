%% Los cap'itulos inician con \chapter{T'itulo}, estos aparecen numerados y
%% se incluyen en el 'indice general.
%%
%% Recuerda que aqu'i ya puedes escribir acentos como: 'a, 'e, 'i, etc.
%% La letra n con tilde es: 'n.

%
%MG: En el resumen debes hablar de todo el trabajo revisa:
%% http://www.slideshare.net/jjmerelo/cmo-escribir-y-publicar-trabajos-cientficos-17948694

% Da una breve introducción al tema (problema)
% Los métodos que pleanteamos aquí son {metodología}
% para [mejorar, evitar, comprobar] {problema}
% Se probó de la manera {x} que los objetivos se alcanzaron
% con los resultados siguientes {x}

% Al integrar estas técnicas el desempeño mejoró en {x} respecto a {y}
% de manera estadísticamente significativa logrando para los datos de prueba
% un error absoluto de XX.
% Es necesario comentar los resultados con el error?..

\prefacesection{Resumen}
La computaci\'on evolutiva interactiva es una rama de la computaci\'on evolutiva
donde la asignaci\'on de aptitud es dado por personas (usuarios). Esta t\'ecnica
est\'a migrando r\'apidamente a aplicaciones Web, debido a la gran cantidad de
usuarios que se encuentran navegando en internet. Uno de los campos de
investigaci\'on donde m\'as  se ha implementado  esta t\'ecnica es en el arte
evolutivo con aplicaciones Web. Normalmente en este campo las aplicaciones
implementadas est\'an utilizando tecnolog\'ias Web, esto con el fin de poder llegar
a una gran cantidad posible de personas, ya que estas  dependen de la
participaciones de personas que visitan la aplicaci\'on de manera voluntaria. Sin
embargo existen algunas barreras que limitan la participaci\'on de las personas,
por ejemplo la fatiga y aburrimiento debido a la gran carga de evaluaciones de
fenotipos (individuos). En esta investigaci\'on proponemos elevar la participaci\'on
de la personas en contexto de computaci\'on evolutivo interactive basadas en Web
mediante el uso de un modelado de usuario basado en grafos, as\'i como tambi\'en el
uso del paradigma de l\'ogica difusa y la competencia basada en videojuegos que es
una t\'ecnica de usabilidad. Lo antes mencionado forma parte de un marco de
trabajo centrado en los usuarios, el cual se implementa en aplicaciones de este
contexto.  As\'i mismo se presenta un caso de estudio el cual provee detalles de
aspectos conceptuales y de implementaci\'on, para abordar la problem\'atica de
participaci\'on de personas dentro de estos sistemas.



\prefacesection{Abstract}
