\chapter{Results} \label{sec:6}
In this chapter the results of the experiments described above are presented.


\section{EvoDrawing01}

The experiment EvoDrawing01 generated the following data shown in the table x


\begin{table}
\small
\caption{Data generated in graph-based user modeling.}
\label{tab:dataGenerated_1} 
\centering
\small
\begin{tabular}{p{3cm} p{3cm} p{3cm} }
\hline\noalign{\smallskip}
  & Data &  \\
\noalign{\smallskip}\hline\noalign{\smallskip}
\small{Nodes} & \small{595} & \\ \hline  
\small{Relationships} & \small{2220} & \\ \hline  
  
\noalign{\smallskip}\hline
\end{tabular}
\end{table}


These data are generated total of nodes as well as relationships.

\begin{table}
\small
\caption{Total number of volunteers active users.}
\label{tab:totalUsers_1} 
\centering
\small
\begin{tabular}{p{3cm} p{3cm} p{3cm} }
\hline\noalign{\smallskip}
  & Users &  \\
\noalign{\smallskip}\hline\noalign{\smallskip}
\small{Total } & \small{53} & \\ \hline    
\noalign{\smallskip}\hline
\end{tabular}
\end{table}

In order to observe which users have better social interconnectivity within the
experiments it was decided to make a relation of number known users that the
user’s have. This relation is shown in table 3

\begin{table}
\small
\caption{Number of known among users.}
\label{tab:knownUsers_1} 
\centering
\small
\begin{tabular}{p{3cm} p{3cm}  }
\hline\noalign{\smallskip}
 User name & Number of known users \\
\noalign{\smallskip}\hline\noalign{\smallskip}
\small{Chriss de Blanc } & \small{7}  \\ \hline 
\small{Alejandro Salcido } & \small{4}  \\ \hline
\small{Jonathan Amezcua Aguiluz } & \small{1}  \\ \hline
\small{Daniela Sanchez } & \small{1}  \\ \hline
\small{Jennifer Llamas} & \small{1}  \\ \hline
\small{Iliana Dl } & \small{1}  \\ \hline
\small{Xochilt Ramirez Garcia } & \small{1}  \\ \hline
\small{Manuel Elizondo } & \small{1}  \\ \hline 
\small{Julian Torres } & \small{1}  \\ \hline 
\small{Data Back } & \small{1}  \\ \hline 
\small{Frank Arce } & \small{1}  \\ \hline 
\small{Gustavo Vargas } & \small{1}  \\ \hline 
\noalign{\smallskip}\hline
\end{tabular}
\end{table}

Likewise in Table 3 interconnectivity having a particular user with other users
is shown. The degree of relationship of users is associated with the number of
friends known within the application. This means that we have the degree of
influence among participants. For example the "Chriss Blanc" user has a degree
of relatedness 7 (Figure x A)) indicating that this particular user can have
more influence on the decisions of others. Moreover the user "Alejandro Salcido"
has the second highest degree of relationship to other users (Figure x B)).

\begin{figure*}
%\captionsetup{justification=centering,margin=1cm}
\centering
\fbox{\includegraphics[scale=0.75]{img/users_known_1.PNG}} %[width=0.7\textwidth]
\caption{Graph users with greater interconnectivity.}
\label{fig:guserknown_1}   
\end{figure*}

In figure \ref{fig:guserknown_1} shows users more connected in the graph. This may represent the degree
of impact of a user and the possible influence that may have on the decisions of
other users. For example, if the user with the greatest impact is affected in
any decision likely other users may be affected in some way.


\begin{table}
\small
\caption{Total number of individuals generated.}
\label{tab:totalIndividuals_1} 
\centering
\small
\begin{tabular}{p{3cm} p{3cm} p{3cm} }
\hline\noalign{\smallskip}
  & Individuals &  \\
\noalign{\smallskip}\hline\noalign{\smallskip}
\small{Total } & \small{500} & \\ \hline    
\noalign{\smallskip}\hline
\end{tabular}
\end{table}


Table 4 presents the total number of individuals generated within the graph.

\begin{table}
\small
\caption{Sample of 30 individuals evaluated from.}
\label{tab:totalIndividuals_1} 
\centering
\small
\begin{tabular}{p{3cm} p{4cm} p{3cm} p{3cm}}
\hline\noalign{\smallskip}
 id & Chromosome & Views & Likes  \\
\noalign{\smallskip}\hline\noalign{\smallskip}
\small{pop:individual:107} & \small{[125, 30, 0, 1, 0, 0, 3, 0, 1, 0, 0, 0, 0, 2, 1]} 
& \small{20} & \small{20}\\ \hline 
\small{pop:individual:133} & \small{[143, 15, 1, 1, 0, 1, 3, 0, 1, 0, 2, 0, 0, 1, 1]} 
& \small{15} & \small{15}\\ \hline 
\small{pop:individual:109} & \small{[125, 30, 1, 1, 0, 0, 3, 0, 1, 0, 0, 0, 0, 0, 1]} 
& \small{15} & \small{15}\\ \hline 
\small{pop:individual:37} & \small{[87, 64, 1, 1, 1, 1, 4, 0, 1, 0, 0, 0, 1, 2, 3]} 
& \small{14} & \small{14}\\ \hline 
\small{pop:individual:36} & \small{[95, 71, 0, 1, 1, 0, 3, 1, 0, 0, 0, 1, 1, 1, 2]} 
& \small{14} & \small{14}\\ \hline 
\small{pop:individual:215} & \small{[138, 29, 1, 1, 0, 1, 3, 0, 1, 1, 1, 0, 0, 1, 1]} 
& \small{13} & \small{13}\\ \hline 
\small{pop:individual:228} & \small{[42, 58, 0, 0, 1, 1, 4, 0, 0, 0, 2, 0, 0, 0, 3]} 
& \small{12} & \small{12}\\ \hline 
\small{pop:individual:48} & \small{[51, 73, 0, 0, 0, 0, 2, 0, 0, 1, 3, 0, 1, 0, 3]} 
& \small{12} & \small{12}\\ \hline 
\small{pop:individual:39} & \small{[60, 12, 1, 1, 1, 1, 4, 1, 1, 1, 0, 1, 0, 1, 1]} 
& \small{12} & \small{12}\\ \hline 
\small{pop:individual:94} & \small{[49, 71, 0, 0, 1, 0, 4, 1, 0, 0, 1, 1, 1, 0, 2]} 
& \small{11} & \small{11}\\ \hline 
\small{pop:individual:194} & \small{[87, 64, 0, 1, 1, 1, 1, 3, 0, 0, 0, 0, 1, 2, 3]} 
& \small{11} & \small{11}\\ \hline 
\small{pop:individual:75} & \small{[97, 66, 0, 0, 1, 1, 3, 0, 1, 0, 3, 0, 1, 0, 1]} 
& \small{11} & \small{11}\\ \hline 
\small{pop:individual:105} & \small{[125, 30, 0, 1, 0, 0, 3, 0, 1, 0, 0, 1, 0, 0, 1]} 
& \small{11} & \small{11}\\ \hline 
\small{pop:individual:306} & \small{[87, 64, 1, 1, 1, 1, 4, 1, 1, 1, 1, 0, 1, 2, 3]} 
& \small{13} & \small{10}\\ \hline 
\small{pop:individual:326} & \small{[138, 29, 1, 1, 0, 1, 3, 0, 1, 1, 1, 0, 0, 0, 0]} 
& \small{10} & \small{10}\\ \hline 
\small{pop:individual:82} & \small{[81, 8, 1, 0, 0, 1, 4, 1, 1, 1, 2, 0, 0, 2, 1]} 
& \small{10} & \small{10}\\ \hline 
\small{pop:individual:252} & \small{[125, 30, 0, 1, 0, 0, 3, 0, 1, 1, 3, 0, 1, 0, 3]} 
& \small{10} & \small{10}\\ \hline 
\small{pop:individual:280} & \small{[53, 63, 1, 1, 0, 1, 3, 0, 1, 1, 0, 0, 0, 2, 1]} 
& \small{9} & \small{9}\\ \hline 
\small{pop:individual:178} & \small{[53, 63, 1, 1, 0, 1, 3, 0, 1, 1, 0, 0, 0, 2, 1]} 
& \small{9} & \small{9}\\ \hline 
\small{pop:individual:108} & \small{[42, 58, 0, 0, 1, 1, 3, 0, 1, 0, 1, 1, 0, 0, 1]} 
& \small{10} & \small{9}\\ \hline 
\small{pop:individual:231} & \small{[125, 30, 0, 1, 0, 0, 3, 0, 1, 1, 3, 0, 1, 0, 3]} 
& \small{0} & \small{9}\\ \hline 
\small{pop:individual:147} & \small{[122, 38, 1, 0, 0, 0, 0, 3, 0, 1, 0, 0, 0, 0, 0]} 
& \small{9} & \small{9}\\ \hline 
\small{pop:individual:181} & \small{[143, 15, 1, 1, 0, 1, 3, 0, 1, 0, 0, 0, 0, 0, 1]} 
& \small{9} & \small{9}\\ \hline 
\small{pop:individual:151} & \small{[49, 27, 1, 0, 0, 0, 3, 0, 0, 1, 2, 0, 1, 0, 3]} 
& \small{9} & \small{9}\\ \hline 
\small{pop:individual:34} & \small{[125, 30, 0, 1, 0, 0, 3, 0, 1, 0, 0, 0, 0, 2, 1]} 
& \small{8} & \small{8}\\ \hline 
\small{pop:individual:185} & \small{[122, 38, 0, 0, 0, 1, 1, 1, 1, 1, 3, 0, 1, 1, 1]} 
& \small{8} & \small{8}\\ \hline 
\small{pop:individual:88} & \small{[49, 62, 0, 1, 1, 1, 1, 1, 1, 1, 0, 0, 0, 2, 2]} 
& \small{8} & \small{8}\\ \hline 
\small{pop:individual:176} & \small{[125, 30, 0, 1, 0, 0, 3, 0, 1, 1, 3, 0, 1, 0, 3]} 
& \small{8} & \small{8}\\ \hline 
\small{pop:individual:234} & \small{[42, 58, 0, 0, 1, 1, 3, 0, 1, 0, 1, 0, 0, 0, 1]} 
& \small{8} & \small{8}\\ \hline    
\noalign{\smallskip}\hline
\end{tabular}
\end{table}

Table 5 contains a sample of 30/500 individuals were generated in the
experiment. Where we present the unique identifier of the individual, its
chromosome, as well as the number of views, likes available to the individual.
This results are useful to observe individuals have been better evaluated by
users.

\begin{table}
\small
\caption{Level of user participation.}
\label{tab:userParticipation_1} 
\centering
\small
\begin{tabular}{p{4cm} p{4cm}}
\hline\noalign{\smallskip}
 User name & Participation   \\
\noalign{\smallskip}\hline\noalign{\smallskip}
\small{Ana Laura Lopez} & \small{116} \\ \hline 
\small{Mario García Valdez} & \small{100} \\ \hline 
\small{Chriss de Blanc} & \small{93} \\ \hline 
\small{Xochilt Ramirez Garcia} & \small{85} \\ \hline 
\small{Carlos David Gallardo Pérez} & \small{73} \\ \hline 
\small{Ulises Reus} & \small{70} \\ \hline 
\small{Aaron Gutierrez Urbina} & \small{58} \\ \hline 
\small{Cesar López} & \small{49} \\ \hline 
\small{Hector Beltran Medrano} & \small{48} \\ \hline 
\small{Luis Alfonso Felix Garcia} & \small{45} \\ \hline 
\small{Data Back} & \small{39} \\ \hline 
\small{Amaury Hernandez Aguila} & \small{32} \\ \hline 
\small{Osmar Herrera Duran} & \small{31} \\ \hline
\small{Jorman Gtz} & \small{29} \\ \hline
\small{Alexis Campos Lopez} & \small{29} \\ \hline
\small{Melissa Muñoz Montes} & \small{28} \\ \hline
\small{David Gallegos} & \small{23} \\ \hline
\small{Jose Carlos} & \small{21} \\ \hline
\small{Tomás Perrín} & \small{21} \\ \hline
\small{Manuel Elizondo} & \small{20} \\ \hline

    
\noalign{\smallskip}\hline
\end{tabular}
\end{table}




Table 6 shows the results of the level of user participation in the experiment.
These were obtained by counting the vicinity of nearest nodes from the base node
in this case each user node.



\begin{figure*}
%\captionsetup{justification=centering,margin=1cm}
\centering
\fbox{\includegraphics[scale=0.75]{img/Visual_represntation_1.PNG}} %[width=0.7\textwidth]
\caption{Visual representation of user participation in EvoDrawing01.}
\label{fig:userP_1}   
\end{figure*}

In figure \ref{fig:userP_1} we present a visual representation of user
participation this experiment where the y-axis represents the level of
participation and the x axis represents the number of users who participated in
this experiment.

\begin{figure*}
%\captionsetup{justification=centering,margin=1cm}
\centering
\fbox{\includegraphics[scale=0.75]{img/weibull_1.PNG}} %[width=0.7\textwidth]
\caption{Weibull fit data representation.}
\label{fig:weibull_1}   
\end{figure*}


\section {EvoDrawing02}


In the experiment EvoDrawings02 the following data were generated presented in Table x.


\begin{table}
\small
\caption{Data generated in graph-based user modeling.}
\label{tab:dataGenerated_2} 
\centering
\small
\begin{tabular}{p{3cm} p{3cm} p{3cm} }
\hline\noalign{\smallskip}
  & Data &  \\
\noalign{\smallskip}\hline\noalign{\smallskip}
\small{Nodes} & \small{648} & \\ \hline  
\small{Relationships} & \small{2596} & \\ \hline  
  
\noalign{\smallskip}\hline
\end{tabular}
\end{table}

These data are generated total of nodes as well as relationships.

The total number of users who participate voluntarily shown in the table x.

\begin{table}
\small
\caption{Total number of volunteers active users.}
\label{tab:totalUsers_1} 
\centering
\small
\begin{tabular}{p{3cm} p{3cm} p{3cm} }
\hline\noalign{\smallskip}
  & Users &  \\
\noalign{\smallskip}\hline\noalign{\smallskip}
\small{Total } & \small{54} & \\ \hline    
\noalign{\smallskip}\hline
\end{tabular}
\end{table}

Like the previous experiment which wanted to observe users had better social
interconnectivity within this experiment, it was decided to make a relationship
of the number of known users. This relationship presented in table x where the
top 10 user  with social interconnectivity are shown.


\begin{table}
\small
\caption{A sample of the top 10 users level influence on users.}
\label{tab:knownUsers_2} 
\centering
\small
\begin{tabular}{p{3cm} p{3cm}  }
\hline\noalign{\smallskip}
 User name & Number of known users \\
\noalign{\smallskip}\hline\noalign{\smallskip}
\small{Rogelio UR} & \small{10}  \\ \hline 
\small{Barbara Sandoval} & \small{9}  \\ \hline
\small{Evelyn Macedo} & \small{8}  \\ \hline
\small{Chriss de Blanc} & \small{8}  \\ \hline
\small{Cesar Rojas} & \small{8}  \\ \hline
\small{Jasiel Calzada} & \small{8}  \\ \hline
\small{Tonyy Maldonado} & \small{7}  \\ \hline
\small{Silvano Peraza} & \small{7}  \\ \hline 
\small{Hector Beltran Medrano} & \small{6}  \\ \hline 
\small{Juan Ferman Lopez} & \small{6}  \\ \hline 
\noalign{\smallskip}\hline
\end{tabular}
\end{table}

In figure 3 shows users more connected in the graph. This may represent the
degree of impact of a user and the possible influence that may have on the
decisions of other users.  For example, if the user with the greatest impact is
affected in any decision likely other users may be affected in some way.
Particularly in this experiment the degree of social interconnectivity among
users is becoming more complex as compared to the previous experiment.

\begin{figure*}
%\captionsetup{justification=centering,margin=1cm}
\centering
\fbox{\includegraphics[scale=0.75]{img/user_known_2.PNG}} %[width=0.7\textwidth]
\caption{Sample of the top 30 individuals evaluated from a total of 556.}
\label{fig:bestIndividuals_2}   
\end{figure*}

\begin{table}
\small
\caption{Sample of 30 individuals evaluated from.}
\label{tab:totalIndividuals_1} 
\centering
\small
\begin{tabular}{p{3cm} p{4cm} p{3cm} p{3cm}}
\hline\noalign{\smallskip}
 id & Chromosome & Views & Likes  \\
\noalign{\smallskip}\hline\noalign{\smallskip}
\small{pop:individual:55} & \small{[63, 58, 0, 1, 1, 1, 4, 0, 0, 1, 0, 0, 0, 0, 3]} 
& \small{18} & \small{18}\\ \hline 
\small{pop:individual:329} & \small{[98, 37, 0, 1, 1, 0, 4, 0, 0, 1, 0, 0, 0, 0, 2]} 
& \small{17} & \small{17}\\ \hline 
\small{pop:individual:304} & \small{[63, 58, 0, 1, 1, 1, 4, 0, 0, 0, 3, 0, 0, 2, 2]} 
& \small{16} & \small{16}\\ \hline 
\small{pop:individual:202} & \small{[107, 79, 1, 0, 1, 1, 3, 0, 0, 1, 3, 0, 1, 1]} 
& \small{15} & \small{15}\\ \hline 
\small{pop:individual:58} & \small{[150, 79, 1, 0, 1, 1, 3, 0, 0, 1, 3, 0, 1, 1, 1]} 
& \small{14} & \small{14}\\ \hline 
\small{pop:individual:310} & \small{[63, 58, 0, 1, 1, 1, 4, 0, 0, 1, 0, 0, 1, 2, 2]} 
& \small{13} & \small{13}\\ \hline 
\small{pop:individual:67} & \small{[65, 51, 1, 1, 0, 0, 3, 1, 0, 0, 3, 1, 0, 2, 3]} 
& \small{12} & \small{12}\\ \hline 
\small{pop:individual:48} & \small{[51, 73, 0, 0, 0, 0, 2, 0, 0, 1, 3, 0, 1, 0, 3]} 
& \small{12} & \small{12}\\ \hline 
\small{pop:individual:179} & \small{[98, 37, 0, 1, 1, 1, 4, 0, 0, 1, 0, 0, 0, 0, 3]} 
& \small{12} & \small{12}\\ \hline 
\small{pop:individual:114} & \small{[63, 58, 0, 1, 1, 1, 4, 0, 0, 1, 0, 0, 0, 0, 2]} 
& \small{12} & \small{12}\\ \hline 
\small{pop:individual:123} & \small{[124, 42, 0, 1, 1, 1, 4, 1, 0, 1, 1, 0, 0, 2, 1]} 
& \small{12} & \small{12}\\ \hline 
\small{pop:individual:344} & \small{[97, 66, 0, 0, 1, 1, 3, 0, 1, 0, 3, 0, 1, 0, 1]} 
& \small{11} & \small{11}\\ \hline 
\small{pop:individual:105} & \small{[63, 58, 0, 1, 1, 1, 4, 0, 0, 1, 0, 0, 0, 0, 2]} 
& \small{11} & \small{11}\\ \hline 
\small{pop:individual:435} & \small{[98, 37, 0, 1, 1, 0, 1, 1, 4, 1, 0, 0, 0, 0, 2]} 
& \small{11} & \small{11}\\ \hline 
\small{pop:individual:140} & \small{[150, 79, 1, 0, 1, 1, 3, 1, 1, 0, 1, 1, 1, 2, 3]} 
& \small{11} & \small{11}\\ \hline 
\small{pop:individual:216} & \small{[124, 42, 0, 1, 0, 0, 3, 0, 0, 1, 1, 0, 0, 1, 1]} 
& \small{11} & \small{11}\\ \hline 
\small{pop:individual:290} & \small{[150, 79, 1, 0, 1, 1, 3, 0, 0, 1, 3, 0, 1, 0, 1]} 
& \small{10} & \small{10}\\ \hline 
\small{pop:individual:255} & \small{[150, 79, 1, 0, 1, 1, 3, 1, 1, 1, 0, 0, 1, 2, 2]} 
& \small{10} & \small{10}\\ \hline 
\small{pop:individual:14} & \small{[89, 66, 0, 0, 1, 1, 3, 0, 0, 0, 2, 0, 0, 2, 1]} 
& \small{10} & \small{10}\\ \hline 
\small{pop:individual:366} & \small{[128, 66, 0, 1, 1, 4, 1, 0, 1, 0, 0, 0, 1, 2, 2]} 
& \small{10} & \small{10}\\ \hline 
\small{pop:individual:215} & \small{[54, 72, 0, 1, 1, 1, 4, 0, 0, 1, 0, 0, 0, 0, 2]} 
& \small{9} & \small{9}\\ \hline 
\small{pop:individual:411} & \small{[113, 58, 0, 1, 1, 1, 4, 0, 0, 1, 0, 0, 1, 2, 2]} 
& \small{9} & \small{9}\\ \hline 
\small{pop:individual:254} & \small{[113, 58, 0, 1, 1, 4, 1, 0, 4, 1, 1, 0, 0, 3]} 
& \small{11} & \small{9}\\ \hline 
\small{pop:individual:285} & \small{[54, 58, 0, 0, 0, 1, 1, 4, 1, 1, 0, 0, 1, 2, 2]} 
& \small{8} & \small{8}\\ \hline 
\small{pop:individual:500} & \small{[63, 58, 1, 1, 1, 0, 3, 0, 1, 1, 0, 0, 1, 2, 0]} 
& \small{8} & \small{8}\\ \hline 
\small{pop:individual:415} & \small{[54, 72, 0, 1, 1, 1, 4, 1, 4, 1, 3, 0, 1, 1]} 
& \small{8} & \small{8}\\ \hline 
\small{pop:individual:300} & \small{[54, 58, 1, 0, 1, 1, 1, 4, 1, 1, 0, 0, 1, 2, 2]} 
& \small{8} & \small{8}\\ \hline 
\small{pop:individual:32} & \small{[113, 29, 0, 0, 0, 0, 4, 1, 0, 1, 0, 1, 1, 1, 3]} 
& \small{8} & \small{8}\\ \hline 
\small{pop:individual:237} & \small{[63, 58, 1, 1, 0, 1, 1, 4, 1, 0, 0, 0, 1, 2, 3]} 
& \small{8} & \small{8}\\ \hline  
\small{pop:individual:268} & \small{[98, 37, 0, 1, 1, 1, 4, 0, 0, 1, 0, 0, 0, 0]} 
& \small{8} & \small{8}\\ \hline   
\noalign{\smallskip}\hline
\end{tabular}
\end{table}

Table 9 contains a sample of individuals 30/556 generated in the experiment.This
table shows the unique identifier of the individual, its chromosome, as well as
the number of views, likes available to the individual.This results are useful
to observe individuals have been better evaluated by users.

\begin{table}
\small
\caption{Level of user participation.}
\label{tab:userParticipation_2} 
\centering
\small
\begin{tabular}{p{4cm} p{4cm}}
\hline\noalign{\smallskip}
 User name & Participation   \\
\noalign{\smallskip}\hline\noalign{\smallskip}
\small{1122212314475816} & \small{122} \\ \hline 
\small{1107674982600275} & \small{91} \\ \hline 
\small{10207544086753085} & \small{86} \\ \hline 
\small{1128990193799346} & \small{85} \\ \hline 
\small{1067084180030552} & \small{83} \\ \hline 
\small{985591718197586} & \small{77} \\ \hline 
\small{969507913124553} & \small{72} \\ \hline 
\small{10207004677610003} & \small{63} \\ \hline 
\small{1223229694371825} & \small{53} \\ \hline 
\small{10153904046011462} & \small{47} \\ \hline 
\small{1032494570125948} & \small{44} \\ \hline 
\small{995610090523549} & \small{42} \\ \hline 
\small{975038365907627} & \small{35} \\ \hline
\small{10207487295119454} & \small{34} \\ \hline
\small{1041989022528624} & \small{32} \\ \hline
\small{471974623003503} & \small{32} \\ \hline
\small{1275844349097672} & \small{31} \\ \hline
\small{978744228875903} & \small{31} \\ \hline
\small{10205734318020434} & \small{31} \\ \hline
\small{10209454397419860} & \small{29} \\ \hline

    
\noalign{\smallskip}\hline
\end{tabular}
\end{table}






In table 10 shows the results of the level of user participation in the
experiment.These were obtained by counting the vicinity of nearest nodes from
the base node in this case each user node.

\begin{figure*}
%\captionsetup{justification=centering,margin=1cm}
\centering
\fbox{\includegraphics[scale=0.75]{img/visual_representation_2.PNG}} %[width=0.7\textwidth]
\caption{Visual representation of user participation in EvoDrawing01.}
\label{fig:userP_2}   
\end{figure*}


In Figure 3 we present a visual representation of user participation of this
experiment where the y-axis represents the level of participation and the x axis
represents the number of users who participated in this experiment.

\begin{figure*}
%\captionsetup{justification=centering,margin=1cm}
\centering
\fbox{\includegraphics[scale=0.75]{img/weibull_2.PNG}} %[width=0.7\textwidth]
\caption{Weibull fit data representation.}
\label{fig:weibull_2}   
\end{figure*}














