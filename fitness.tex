\chapter{Fitness Estination}

This chapter explains a proposed strategy to calculate the fitness of
individuals in web-based IEC applications
using fuzzy logic. Before showing our strategy, it is necessary to explain how
the individual evaluation is made in the EvoDrawing application. Figure
\ref{fig:UI_ED} shows the user interface. 
The main goal of this interaction is the evaluation of individuals, but
the first user action before starting to evaluate, is to login through the
Facebook \cite{facebook} social
platform account. The evaluation takes place through a
five star rating selection; this rate represents the degree of user
preference for an individual. The application keeps record of every user
activity by using the activity stream standard [14]. %%%%% Fix reference - Mario
In these particular case activities represent the user\'s experience.

\begin{figure*}
\captionsetup{justification=centering,margin=2cm}
\centering
\setlength\fboxsep{0pt}
\setlength\fboxrule{0.7pt}
\fbox{\includegraphics[width=12cm,height=10cm,keepaspectratio]{img/UI_ed01.png}}
\caption{User interface ED01.}
\label{fig:UI_ED}
\end{figure*}

In this research we propose the use of fuzzy logic[10] in order to obtain a %% Fix references please - Mario
difuzzify value to be used to calculate the individual fitness trough a fitness
function expression. It is used by modeling a fuzzy Mamdani type inference

system [5] [6] as figure \ref{fig:fis01} shown. This model was designed empirically. %% Fix references please - Mario
The model consists of two input variables, which are the preference and the
experience of the user as well as an output that we called fuzzy rate. The first
one has 3 linguistic variables, which are low, medium and high, representing the 
% These are not variables. Low is a linguistic variable?? 
preference with triangular membership functions over a range of 1 to 5. The
second one also has three linguistic variables, which are low, medium and high
representing the experience with triangular membership functions over a range of
1 to 100. Finally we have the output consisting of three linguistic variables
bad, normal and good representing the fuzzy rate with triangular membership
functions in a range of 1-100.

\begin{figure*}
\captionsetup{justification=centering,margin=2cm}
\centering
\setlength\fboxsep{0pt}
\setlength\fboxrule{0.7pt}
\fbox{\includegraphics[width=12cm,height=10cm,keepaspectratio]{img/fuzzy_system_2_1.png}}
\caption{Fuzzy System Mamdani Type.}
\label{fig:fis01}
\end{figure*}

Below we show the rules IF-THEN of the fuzzy system:

\begin{enumerate}
	\item \textit{If \textbf{preference} is low and
		\textbf{experience} is low then \textbf{fuzzy\_rate} is bad.}
	\item \textit{If \textbf{preference} is mid and
		\textbf{experience} is low  then \textbf{fuzzy\_rate} is bad.}
	\item \textit{If \textbf{preference} is high and
		\textbf{experience} is low  then \textbf{fuzzy\_rate} is normal.}
	\item \textit{If \textbf{preference} is low and
		\textbf{experience} is mid then \textbf{fuzzy\_rate} is bad.}
	\item \textit{If \textbf{preference} is mid and
		\textbf{experience} is mid  then \textbf{fuzzy\_rate} is normal.}
	\item \textit{If \textbf{preference} is high and
		\textbf{experience} is mid  then \textbf{fuzzy\_rate} is good.}
	\item \textit{If \textbf{preference} is low and
		\textbf{experience} is high then \textbf{fuzzy\_rate} is normal.}
	\item \textit{If \textbf{preference} is mid and
		\textbf{experience} is high  then \textbf{fuzzy\_rate} is good.}
	\item \textit{If \textbf{preference} is high and
		\textbf{experience} is high  then \textbf{fuzzy\_rate} is good.}

\end{enumerate}

These rules will give us a fuzzy rate value as a result, this value needs to
be defuzzified by the centroid method in order to be used in our fitness
expression, given by the equation\ref{eq:fitfunc_02}. This expression is responsible
of representing the individual fitness.

\begin{equation}\label{eq:fitfunc_02}
\displaystyle fitness=\frac{\sum_{i=0}^{n}x_{i}+f(y_{i})}{\sum_{i=0}^{n}f(y_{i})}
\end{equation}

Where $n$ represents  the number of  users that have evaluated the
individual, $x$ is the rate of preference for the  individual  given by the user,
$y$ is a function that calls the fuzzy system in order to have the fuzzy rate.
This function has as a parameter the rate $x$ and the user experience level. 
The user experience level
is given  by the  total activities that user has at the moment. In each
activity we assign the score, for example if the user log in (join) to the
application we assign 5 points, if the user evaluates (likes) an individual we
give 3 points, etc; in figure 5 shows the flow for assigning fitness to
the individual. 

