\documentclass[oneside,numbers,english, nocolors ]{./ezthesis}
%\usepackage{subfig}
%\usepackage{multirow}
\usepackage{graphicx,rotating,booktabs}
\usepackage{lscape}
%\usepackage[spanish]{babel}
\usepackage[spanish,USenglish]{babel} % espanol, ingles
%\usepackage[utf8]{inputenc} % acentos sin codigo
%\renewcommand{\chaptername}{chapter}
%\renewcommand{\tablename}{Table}
%\renewcommand{\figurename}{Figure}
\usepackage{multirow, array} %para mover el ancho de las tablas.
\renewcommand{\baselinestretch}{2}
%% # Datos del documento #
%% Nota que los acentos se deben escribir: \'a, \'e, \'i, etc.
%% La letra n con tilde es: \~n.
%0.15
\usepackage{vmargin}
\setpapersize{USletter}
\setmargins{3.2cm}       % margen izquierdo
{2.5cm}                        % margen superior
{15cm}                      % anchura del texto
{20cm}                    % altura del texto
{10pt}                           % altura de los encabezados
{1cm}                           % espacio entre el texto y los encabezados
{10pt}                             % altura del pie de página
{1cm}                           % espacio entre el texto y el pie de página


\usepackage{caption}
\usepackage{subfig}
\usepackage{quotchap}

\usepackage{amssymb, amsmath, amsbsy} % simbolitos
\usepackage{upgreek} % para poner letras griegas sin cursiva
\usepackage{cancel} % para tachar
\usepackage{mathdots} % para el comando \iddots
\usepackage{mathrsfs} % para formato de letra
\usepackage{stackrel} % para el comando \stackbin

\usepackage{algorithm}
\usepackage{algorithmic}
\usepackage{listings}
%\usepackage{algpseudocode}
%\usepackage{pifont}
\usepackage[numbers]{natbib}
\usepackage[verbose]{placeins}
%\usepackage[ngerman]{babel}
\usepackage{blindtext}

\hyperlinking
% \bibliographystyle{ieeetr}
\begin{document}

\pagenumbering{Roman}
%\include{front-thesis}


\prefacesection{Resumen}

La intervenci\'on necesaria de los seres humanos en sistemas de computaci\'on
evolutiva interactiva tiene inconvenientes inherentes que surgen de la
naturaleza misma de los algoritmos, como por ejemplo la fatiga causada por la
interacci\'on y el aburrimiento que surge cuando los usuarios eval\'uan un gran
n\'umero de artefactos. Para abordar estos problemas, en esta tesis se propone
un marco de trabajo centrado en el ser humano para
modelar las interacciones complejas de estos sistemas. Se presenta un caso de
estudio donde el modelo se aplica en el desarrollo de un sistema evolutivo
interactivo. Ambos detalles de implementaci\'on como conceptuales
son expuestos para medir e incrementar la participaci\'on de los usuarios
participantes. Los experimentos demuestran que el modelo puede aplicar con
\'exito la t\'ecnica de competencia basadas en videojuegos (gamification) para
desarrollar el aumento de atenci\'on de los usuarios, la cual implica que esta
t\'ecnica puede ser utilizada para disminuir la fatiga y el aburrimiento de los
usuarios, y as\'i aumentar el rendimiento del sistema evolutivo interactivo.




\prefacesection{Abstract}

The necessary intervention of humans in interactive evolutionary computational
systems has inherent drawbacks arising from the very nature of the algorithms,
namely the human fatigue caused by the interaction, and the boredom arising
when users evaluate a large number of individuals. To tackle these issues, in this
thesis a human-centered framework is proposed to model complex interactions in
these  systems. A case study is presented, where the model is applied in the
development of an interactive evolutionary system. Both conceptual
and implementation details are provided where the technique is used to measure
and increase user engagement and participation. Our experiments show that the
model can be successfully applied in a gamification technique developed to
increase user engagement, which implies that this technique can be successfully
used to decrease user fatigue and boredom, and thus increase the performance of
the interactive system.

%% Los cap'itulos inician con \chapter{T'itulo}, estos aparecen numerados y
%% se incluyen en el 'indice general.
%%
%% Recuerda que aqu'i ya puedes escribir acentos como: 'a, 'e, 'i, etc.
%% La letra n con tilde es: 'n.

% 
%MG: En el resumen debes hablar de todo el trabajo revisa: 
%% http://www.slideshare.net/jjmerelo/cmo-escribir-y-publicar-trabajos-cientficos-17948694

% Da una breve introducción al tema (problema)
% Los métodos que pleanteamos aquí son {metodología}
% para [mejorar, evitar, comprobar] {problema}  
% Se probó de la manera {x} que los objetivos se alcanzaron  
% con los resultados siguientes {x} 

% Al integrar estas técnicas el desempeño mejoró en {x} respecto a {y}
% de manera estadísticamente significativa logrando para los datos de prueba 
% un error absoluto de XX.
% Es necesario comentar los resultados con el error?.. 

\prefacesection{Dedicatoria}



Dedico este trabajo a mi esposa Iliana Durón Landero por el apoyo incondicional
que brindaste durante todo el proceso. 
“Sabemos muy poco, y sin embargo es
sorprendente que sepamos tanto, y es todavía mas sorprendente que tan poco
conocimiento nos de tanto poder.” Bertrand Russell        
“Después de todo, cualquier tipo de conocimiento implica auto-conocimiento.” Bruce Lee 
“El conocimiento descansa no solo sobre la verdad sino también sobre el error.” Carl
Jung 
“El conocimiento depende del tiempo, mientras que el saber no. El
conocimiento es una fuente de acumulación, de conclusión, mientras que el saber
es un continuo movimiento.” Bruce Lee












%% Las secciones del "prefacio" inician con el comando \prefacesection{T'itulo}
%% Este tipo de secciones *no* van numeradas, pero s'i aparecen en el 'indice.
%%
%% Si quieres agregar una secci'on que no vaya n'umerada y que *tampoco*
%% aparesca en el 'indice, usa entonces el comando \chapter*{T'itulo}
%%
%% Recuerda que aqu'i ya puedes escribir acentos como: 'a, 'e, 'i, etc.
%% La letra n con tilde es: 'n.

\prefacesection{Agradecimientos} Agradezco al CONACYT por haberme brindado la
oportunidad y el apoyo a trav\'es de la beca 256554 durante el periodo de
Agosto-2011 a Agosto-2015 para concluir los estudios de Doctorado en Ciencias en
Computaci\'on.

A mi director de tesis, Dr. Jos\'e Mario Garc\'ia Valdez  que le tengo una
enorme gratitud por haberme ense\~nado herramientas tan valiosas para mi
formaci\'on como profesional as\'i como persona de tecnolog\'ia. Al Dr. Oscar
Castillo L\'opez y tambi\'en a la Dra. Elba Patricia Mel\'in por haberme
brindado la oportunidad de entrar al programa de doctorado en ciencias de la
computaci\'on del Instituto Tecnol\'ogico de Tijuana, a todos ustedes gracias
por su paciencia y recomendaciones para hacer posible este proyecto de tesis.


A mi familia, por su comprensi\'on y apoyo, en especial a mis padres, el Ing.
Julian Torres Ruiz, Lic. Susana Hern\'andez Velarde por haberme apoyado en el
cuidado de mi hija Victoria Alessandra Romero Dur\'on, ya que sin ustedes hubiera
sido imposible lograr esta meta.



\textit{}


%% # 'Indices y listas de contenido #
%% Quitar los comentarios en las lineas siguientes para obtener listas de
%% figuras y cuadros/tablas.
%\begin{singlespace}
\listoffigures
\listoftables
\tableofcontents
%\end{singlespace}
\clearpage
%\listoffigures
%\listoftables
%% # Cap'itulos #
%% Por cada cap'itulo hay que crear un nuevo archivo e incluirlo aqu'i.
%% Mirar el archivo "intro.tex" para un ejemplo y recomendaciones para
%% escribir.%\setcounter{page}{6}

\pagenumbering{arabic}
\chapter{Introduction} \label{introduction} 


\par It is a reality that the World Wide Web in recent years, is growing
exponentially, which means the presence of millions of users on Web sites, Web
applications, Web systems, etc. []. There is a wide variety of Web systems,
where we have different users interacting with them. These users have different
goals when using these Web systems. For example to serching  in Google
\cite{google} for a particular topic, make a reservation for a room in a luxury
resort, check their bank account or simply checking their Facebook account
status [].This variation of users represents a complex diversity as individuals
[]. This diversity lies in the fact that users have different skills, interests,
preferences, knowlage defferent  ways of thinking and learning [].For this
reason users most to interact with the information presented by existing Web
systems. \par When we intend to customize any element in Web system,  it is
necessary to some personal information about the user. This information is a
collection of needs, characteristics, tastes, preferences among others.  This
information allows designers to build a representation of knowledge about the
users. This is what is known as user modeling (UM).    
\par  In order to
understand specific users a user model must be constructed, this can be simple
as a profile where the basic knowledge recorded. Or it can be as complex as
complete representation of it's characteristics, needs, interests and
preferences. In order to understand specific users. The main goal of user
modeling is to represent aspects of the real world of the user's in an autonomous
way.

\par On the other hand, interactive evolutionary computation (IEC)
is a branch of evolutionary computation where users become a part of the
evolutionary process by replacing the fitness function; evaluating individuals
of a population based on their personal preferences[13]. These evaluations are
subjective according to the user point of view based on their perceptions,
interests and desires.  
\par Normally such systems require users to evaluate large amounts of individuals 
actively, causing them to lose interest in participating by the fatigue that 
is generated[13]. Nowadays some of these systems
are migrating to  Web technologies looking for volunteer users to collaborate
in the evaluation for distribute the load and lower the fatigue. Having Web-
based interactive evolutionary systems opens the possibility of linking to social networks 
platforms in order to involve a larger number of users to assist in
the evaluation of individuals produced by these systems.

\par This research presents a multidisciplanary method that involves several techniques 
such as graph-based user modeling, fuzzy logic and human-interaction in the context 
of a Web-based interactive evolutionary computation. The purpose of this research 
is to increase voluntary participation of the user's  using the proposed method.

This thesis is organized as follows:
\begin{itemize}
\item  \textbf{Chapter 2: State of the art.}{The related work is presented.}
\item  \textbf{Chapter 3: Background.}{The theory and background are preseted.} 
\item  \textbf{Chapter 4: Proposed Method.}{The proposed of the method is described, its characteristics and main objectives. } 
\item  \textbf{Chapter 5: Study Case.}{Description of the scenarios of the experiments that were used in this research.}
\item  \textbf{Chapter 6: Results.}{The experimental results are shown.}
\item  \textbf{Chapter 7: Conclusions and Future Work.}{Conclusions and future work are presented.}
\end{itemize}
\chapter{State of the art} \label{}


Some 1986's Dawkins's research was the pioneer of a significant addition to the
1990s IEC algorithms research works\cite{}[Dawkins 1986].

There is two key research approach about his field:

Creative Approach: The Artificial Life (AL) was the base of creative approach.
AL uses complex algorithms for biological life models emulation. To perform this
task, it is needed to include some of the different techniques starting from
right image treatment. Good graphic creation as well as a great music and
quality sounds, [Sims 1991b], [Sims 1991c], [Sims 1994], [Dawkins 1986], [Disz
1997], [Unemi 2000]and [Unemi 2003].
 
Humanized technology approach: The concept of humanized technology approach
comes from the approach that is focused on the IEC algorithms interface, this is
the research of interaction between humans and computer systems. The main goal
of this was to reduce the user's fatigue and to promote the inputs and outputs
of algorithms to improve the efficiency of them. IEC has made his own way in
practical fields such as engineering, education,etc., [Parmee 1993], [Ventrella
1994a], [Takagi 1996], [Poli 1997], [Parmee 1998] and [Takagi 1998].

Computer graphics (CG) The Biomorph of Dawkins was the first IEC research, from
this research comes to many motivated works mostly about the Selfish Gene, come
of these works are:  [Ochoa 1998], [Mccormack 1993], and [Smith 2003].

In Dawkins work a conventional recursive algorithm was used as a baseline
maintaining the main target of trees with an L-system Lindenmayer.

Some 1986's Dawkins's research was the pioneer of the significant addition to
the 1990s IEC algorithms research works[Dawkins 1986].

There is two key research approach about his field:

Creative Approach: The Artificial Life (AL) was the base of creative approach.
AL uses complex algorithms for biological life models emulation. In order to
perform this task it is needed to include some of the different techniques
starting from right image treatment , good graphic creation as well as a great
music and quality sounds, [Sims 1991b], [Sims 1991c], [Sims 1994], [Dawkins
1986], [Disz 1997], [Unemi 2000]and [Unemi 2003].
 
Humanized technology approach: The concept of humanized technology approach
comes from the approach that is focused on the IEC algorithms interface, this is
the research of interaction between humans and computer systems. The main goal
of this was to reduce the user's fatigue and to promote the inputs and outputs
of algorithms to improve the efficiency of them. IEC has made his own way in
practical fields such as engineering, education,etc., [Parmee 1993], [Ventrella
1994a], [Takagi 1996], [Poli 1997], [Parmee 1998] and [Takagi 1998].


Computer graphics (CG) The Biomorph of Dawkins was the first IEC research, from
this research comes to many motivated works mostly about the Selfish Gene, come
of these works are:  [Ochoa 1998], [Mccormack 1993], and [Smith 2003].

In Dawkins work, a conventional recursive algorithm was used as a baseline
maintaining the main target of trees with an L-system (Lindenmayer). This same
L-system was the base for another experiment to create 2-D CG forms insects from
a system called Blind Watchmaker who used L-system angles from L-system output
intuitively selected; the creation was called biomorphs. These creations reach
his target with the multiple selections of the users based on their preferences;
all these selections acted like a natural adaptation filter.

We can find plenty of applications and works for fractal generation [Sims 1991a]
and [Sims 1992], [Baluja 1993] and [Baluja 1994], [Lund 1995], or [Angeline
1996],[Raynal 1999] and [Lutton 2003], for rendering in tridimensional, [Todd
1991],[Broughton 1997], [Das 1994] and [Tam 2002], for generation of virtual
creatures, [Sims 1994], [Rowland 2000], or aerodynamic surface design (wings),
[NGuyen 1993], [NGuyen 1994] and [NGuyen 1997].

We can discover more than one additional way to use this research in the
artistic field with several applications of IEC who are used for cartoon face
construction and animations matters, like Mutator [Todd 1991], [Todd 1994]
and[Todd 1999] or [Bentley 1999a].

The genetic programming (GP) applications offers a category called Interactive
Genetic Programming (IGP) with many examples of successful application in
tridimensional artwork for artistic animations or construction using
mathematical equations as CAVE [Das 1994], [Papka 1996] and [Disz 1997], [Sims
1991], [Sims 1991a], [Sims 1992],[Sims 1993] and [Min 2004]. As this work
consequence, Panspermia orPrimordial Dance was created.

Imagen

Imagen

The artistic field is only the first step of a great IEC implementation; it is
important to mention another relevant projects called Galapagos, [Sims 1997],
and SBART, [Unemi 2000]. The IEC application Galapagos Project is the exhibit in
Tokio Multimedia Museum, (NTT Intercommunication Center) and this project
originates engaging images to all visitors based on L-systems.

Imagen

There are created after one selection, to get a good solution through multiple
repetitions. This action is performed with Genetic Programming (GP), after the
calculation of each pixel value using trees of equations combining logarithm,
maximum, and minimum, sine, root, cosine, exponential arithmetic operators.
AnimationLab is found as an outstanding work who offer figures that can run or
walk working with the user to receive more opportunities to be picked. A
particular characteristic of all of the figures is that the figures extremities
Mentioning open source works, we can find SBART as an IGP [Unemi 2000] tool to
create graphics. SBART allow to users to evaluate 20 two-dimensional images,
subsequently twenty new image has direction and angles.

Imagen

There are many examples for this field application as [McKenna 1990], [Ventrella
1994a], [Ventrella 1994b], or [Ventrella 1995], [Lim 1999] and [Lim 2000].  One
of the Interactive Evolutionary Programming (IEP)  artistic application was
created by [Angeline 1996], as a fractal generation where the system allows the
evolution of animations for the ones who were selected from the user, the
application initially show only 10 animations to rate.

Music and sound

It is important to know how IEC was implemented in music generation, with
several applications in this field. We will start mentioning the pioneer
application GENJAM, [Biles 1994], [Biles 1996] or [Biles 1999] and [Biles 2000].
Some other attractive works are Sonomorph, [Nelson 1993] and[Nelson 1995], or
SBEAT, [Unemi 2003], [Horowitz 1994], [Onisawa 2000], [Tokui 2000] and [Fels
2002]. It is possible to hear a part of the music songs of these previously
mentioned works broadcasted in the radio station WDYN. (100.1, New York, USA,
WEBPage:http://www.wdyn.net/).

The IEC algorithms are the base for the functionality of the music generation
systems, a visual representation of this is given in the below figure:

Imagen

\begin{sidewaystable}[]
  \caption{Comparison of context-aware recommender systems.}
    \label{tab:stateoftheart}
  \bigskip
    \centering\small\setlength\tabcolsep{2pt}
        \hspace*{-1cm}\begin{tabular}{p{3.5cm} p{6cm} p{4cm} p{3cm} p{3cm} }%{l l l l l}
           \toprule
             \textbf{Application} &\textbf{Contextual Factor} &\textbf{Domain} &\textbf{Paradigm} &\textbf{Device}  \\ \hline

           \midrule
             \textbf{CoMoLE} & \textbf{Time, available time, place, device, level of knowledge, learning style.} & \textbf{E-learning} & \textbf{Pre-filtering} & \textbf{Mobiles, PC, laptop.}   \\ \hline 

             \textbf{Moma-System} & \textbf{Location, time.} & \textbf{E-commerce} & \textbf{Post-filtering} & \textbf{PC, laptop.}  \\ \hline

             \textbf{UbiquITO} & \textbf{Season, time, temperature.} & \textbf{Tourism} & \textbf{Post-filtering} & \textbf{Mobiles} \\ \hline

             \textbf{ReRex} & \textbf{Distance of the point of interest,  temperature, weather, season, weekend, companion, travel goal, transport.} & \textbf{Tourism} & \textbf{Model-based} & \textbf{Mobiles} \\ \hline

             \textbf{LifeTrack} & \textbf{Location, time, day of the week, traifc noise(level), temperature, weather.} & \textbf{Music} & \textbf{ Post-filtering} & \textbf{PC, Mobiles.} \\ \hline

             \textbf{CARS} & \textbf{Location and season.} & \textbf{Restaurants} & \textbf{Post-filtering} & \textbf{PC, laptop.} \\ \hline

             \textbf{InCarMusic} & \textbf{Driving style, road type, landscape, sleepiness, traffic conditions, mood weather and natural phenomena.} & \textbf{Music} & \textbf{Model-based} & \textbf{Mobiles} \\ \hline

            \textbf{REJA} & \textbf{Location.} & \textbf{Restaurants} & \textbf{Pre-filtering and Post-filtering} & \textbf{PC, laptop, mobiles.} \\ \hline

            \textbf{CiberGuide} & \textbf{Location, time, weather.} & \textbf{Tourism} & \textbf{Post-filtering} & \textbf{Mobiles} \\ \hline

            \textbf{PECITAS} & \textbf{Location, routes.} & \textbf{Transport} & \textbf{Post-filtering} & \textbf{Mobiles} \\ \hline

            \textbf{LARS} & \textbf{Tourists’ location and time.} & \textbf{Tourist packages} & \textbf{Post-filtering} & \textbf{Mobiles} \\ \hline

            \textbf{I'm feeling LoCo} & \textbf{Location, transportation.} & \textbf{Tourism} & \textbf{Model-based} & \textbf{Mobiles} \\ \hline

            \textbf{MOPSI} & \textbf{Location} & \textbf{Tourism and transport} & \textbf{Post-filtering} & \textbf{Mobiles} \\ \hline

           \bottomrule
        \end{tabular}\hspace*{-1cm}
\end{sidewaystable}









%%%% La letra n con tilde es: 'n.

\chapter{Background}

This chapter present the fundamental concepts related this work. The
formal definitions referring to fuzzy systems, contextual factors and
recommender system techniques used in the proposed method.
%--------------------------------------------------------------
%Agregue la seccion de logica difusa del articulo que me mando.
\section{User Modeling}

%%\subsection{Traditional Production Systems}

User modeling can be represented as the technique of building a model of the
user to personalize a system. The user model is commonly created as the user is
working with the system. An example is an educational application that teaches
students an individual skill: given the rules and knowledge in the user model,
the difficulty level of the exercises in the form is altered as the user
progresses.   Formally definition of user modeling according to McTear (1993, p.
158): " user modeling is the process of gathering information about the users of
a computer system and of using the information to provide services or
information adapted to the specific requirements of individual users (or groups
of users)". The purpose of the user model is to have a module containing the
operations that are needed to personalize the system, and the user profile,
which includes the personal data of the user ( Mohamad et al., 2013).    System
personalization over user modeling is related to the research field of adaptive
systems; this subject is beyond the scope of this research work. Focus on the
human user, user modeling is a very cross-disciplinary research topic,
comprehending the domains of artificial intelligence, computer science, and
social science. Ideas have been co‐opted from an extensive range of subdomains,
such as human–computer interaction, e‐learning, information science, social
computing, machine learning, data mining, cognitive science, and so on (Kay, et
al., 2012; Kobsa, 2001). There is interest in user modeling from both a
scientific and commercial perspective (Razmerita, 2009).

\subsection{Application Domains for user modeling}

Amount Research and implementation exist in this domain in which personalization
and user modeling plays an important role. This section presents several works
of these domains. To understand this topic, the different objects are divided
into three general categories: 
* supporting a user during a task. 
* giving a user a specific personalized experience. 
* training and educating a user.  
The categories especially differ in the kind of user data that is used. For each
domain, the general purpose of the domain and the more accurate purpose of the
user model are discussed.

\subsection{User models for providing task support}

Task support systems are f systems that help a user during a task by either
supporting the user perform the task or by completely taking over this task
(Nurmi et al., 2007; Brun et al., 2010). For instance,  an application that
automatically categorizes the incoming emails of the user. The goal of the user
model in these requests is to promote the efficiency of interactions with the
user, to simplify these interactions and to make complex systems more usable
(Razmerita, 2009; Fischer,  2001). To perform this personalization, data is
collected through observations of the user. This information is related to the
user’s goals and needs, but especially to the task that the user currently is
accomplished, like the user’s task knowledge and background. Much research has
been done in this domain, but because many separate research projects are
focusing on an exact task or subject (Sannes, 2011), it is hard to make
generalizations or to establish one delimited investigation topic. Commonly
discussed research subjects are Decision Support Systems,  Adaptive Hypermedia,
and Adaptive Ubiquitous Systems, each having his or her own specific domain and
way of personalization.  

\subsection{Decision support systems}

Decision support systems are systems that support a user with making a decision
in a complex, professional environment (Nurmi et al., 2007). For example,  a
system used at a pharmacy for automatically checking valid combinations of
medicine. The method can be used to help the pharmacist in prescribing the right
combinations and to give information for making a decision when a problem
occurs.  The purpose of the user model in decision support systems is to present
the user with the right and appropriate information, giving different feedback
or applying various decision steps according to the characteristics of the
user.The data that is used is often associated with the user’s task and
background knowledge.    The adaptation takes place by adapting the amount and
the content of the feedback provided by the system.


Decision support systems are traditionally ruled ‐ or logic-based systems, in
which all the relevant information is represented in a knowledge base. This
means that the content of the user model itself is also highly dependent on the
way the rules and knowledge are represented.


\subsection{Adaptive Hypermedia}

Adaptive hypermedia system is a system that grant users to browse freely information
network,  structured by nodes and links, to retrieve items of information (Nurmi
et al., 2007; Deepa et al., 2012). For instance an internet website application.
The goal of the user model is to make the interface and structure of the system
dynamic. This enables the application to adapt to the user and to make it easier
for the user to search for and retrieve relevant information.  The data used in
the user model is related to the user’s abilities, knowledge, and goals in the
application. The adaptation happens by adjusting the structure and the
presentation style to the expected needs of the user. For example, by enhancing
web search: promoting pages that might better correspond to the user’s
characteristics, on the other hand by giving navigation support, through
highlighting certain components of a page (Razmerita et al., 2012).

\subsection{Adaptive ubiquitous systems}

Ubiquitous systems are concerned with data handling applications integrated into
everyday objects and activities (Nurmi et al., 2007).  For example the smart
energy meter, recording the energy usage in a household through small devices
distributed in a house, supporting the user with managing this energy
usage(Hargreaves et al., 2010).   
*  The main purpose of the user model is to
improve the system, facilitate the user’s preferences and thus make the overall
use easier.   
* Because the personalization can take place in every situation
and location, the data is focused on the user state and context. For example to
enable the contextualization to a current environmental change. 
*The adaptation
takes place by changing the behavior of, and the feedback given by the whole
system. These objects can be inferred by looking at the properties of the
objects in the user profile, or by looking at the objects in other user profiles
that are similar to the user (Kobsa, 2001; Kay et al., 2012). Because of the
predominantly commercial goal of these systems, the adaptations often take place
in a very intrusive way, to make sure the user notices the change.
Most recommender systems used on the Internet, which means that some typical
technical difficulties are associated with these kinds of user models. First,
the user profile is often only saved during the user’s visit,  which means that
fast and efficient adaption is necessary. Recommender systems often become more
precise when the user spends more time on the system. Second, the system’s
structure is often split up in a client and on a server side, where the client
side solely gathers user information and sends it to the server, where the
actual computation takes place.

\subsection{User models for providing a personal experience.}

User models for providing the user with a personal experience have the goal to
improve the user experience while using the system.  This kind of user modeling
is especially focused on more commercial fields, such as e-commerce, marketing,
and computer games, and became popular with the rise of the Internet.  The
information that is used by the user in this main domain is mostly focused on
the information that defines the user, such as the user’s preferences and
interests. Since this data is regularly delicate, privacy is a  big issue (Toch
et al., 2012). While in other domains the privacy of the user data is also
important, in this area it is even a greater topic of discussion because the
incentive of the application developers is frequently contradictory to the
incentive of the actual user, considering gaining and sharing the user´s
personal information. For instance, user profiles are often shared among diverse
components of the same application, or even with different applications (Brun et
al., 2010; Karam et al., 2012), which presents additional weaknesses and
possible undesirable information sharing.  Ensuring personal data is not open to
all people, in addition to defining strict privacy policies, is thus essential
in these user models. Some investigation in this domain are.

\subsection{Recommender Systems and User Adaptive Computer Games.}

Recommender systems are concerned with presenting the user with relevant
information and suggestions. They are commonly used on the Internet, for example
on websites such as Facebook, to provide the user with personalized news,
targeted advertisements and possibly new friends (Brun et al.,  2010). The
purpose of the user model is to give the system with information that is assumed
to be important for the user. The information that is stored for this goal is
associated with the preferences of the user to certain objects, like products,
music or people. To benefit a classification of these objects, the interaction
history of the user is stored, or the user is explicitly asked to rate certain
objects. The content of the system is eventually adjusted by showing the
recently inferred objects. In these senses, objects can be inferred by looking
at the attributes of the objects in the user profile, or by looking at the
objects in other user profiles that are related to the user (Kobsa, 2001; Kay et
al., 2012).  As a result of the predominantly commercial target of these
systems, the adaptations often take place in a very invasive way, to make sure
the user notices the change. Most recommender systems are based on the Internet,
which means that some typical technical difficulties  are associated with these
kinds of user models. First, the user profile is often only stored while the
user’s visit,  which means that fast and efficient adaption is important.
Recommender systems usually become more precise  when the user spends more time
with the system. Second, the system’s architecture is usually client-server,
where the client side gathers user information and sends it to the server, where
the  actual process takes place.

\subsection{User Adaptive Computer Games.}  

User Adaptive computer games are games that focus on increasing the perceived
value by providing a strongly individualized experience (Brisson, 2012).  For
example is a first‐person shooter that adapts the performance of the enemy
according to the shooting accuracy of the player.

The fundamental idea of the user model is to identify or classify the user, so
the appropriate adjustment is made in the computer game. The information that is
used addresses the preferences and progress of the user, such as the user´s
current  difficulty level or even the employed strategy. This data is usually
obtained through the interactions of  the user with the game, and therefor first
should be translated and formalized  before it can be  used to interpret
conclusions on a higher level. The adaptation that takes place in the game
concerns changing the content and role of the  game, such as the game
difficulty, the behavior of non‐player characters  or even the background  music
(Bakkes, et al., 2012) .

Because of the emphasis on the user, user adaptive computer games have
relatively a lot of processing power available  for personalization. In this
sense the user adaptive computer games domain is a very interesting research
domain.


\subsection{User models for educational purposes.}

Educational systems are systems developed with a teaching reason. They are
commonly applied in e‐learning,  where electronic media and Information
technologies are used for education. However, in most educational systems,  user
modeling and adaptation plays a minor role. Content is presented, and only
simple things such as the student’s progress in the course are registered. By
adding personalization to these applications, the learning value can notably
increase, ensuring that every learner achieves and reaches the highest standards
possible (Heller et al., 2006). Also, the experience of the teacher or
supervisor can be increased through personalization, for instance through
inferring and employing the preferred teaching style. However, here we consider
the student as the user to which the system will be personalized. Thus, it is
preferable that the data stored in the user profile be interpretable by humans.
When looking at the time of adjustment in educational systems, we can make a
clear difference between adaptation while the student is doing an exercise,
which we will refer to as online adaptation, and adaptation that takes place
afterward, which we will refer to as offline adaptation. The most important
investigation domains that do utilize considerable user modeling constructions
are Intelligent  Tutoring Systems and Adaptive Educational Games.

\subsection{Intelligent Tutoring Systems.}

Intelligent Tutoring Systems (ITS) are systems that provide students automated
step‐by‐step instruction as the students complete training tasks and/or work on
exercises. An ITS has the purpose to complement or even replace the human
teacher. For example a system for teaching students how to program, with the
ability to automatically detect common mistakes (Elsom‐Cook, 1993).

The particular goal of the user model is to select educational activities and
strategies and in addition  delivered individual feedback that is most relevant
to the user’s level of knowledge (Kobsa, 2001;  McTear, 1993). The user
information that is stored for this purpose is the student’s state, knowledge
and level of  achievement. This data is exclusively observed over the actions
and results of the student, such as the answers the student gives. After
observing this information, it is used to infer higher level  properties, such
as the student´s learning style and other preferences.

Traditionally, just like decision support systems, information technology
systems are knowledge based systems, using formalized domain  knowledge and
rules to drive the user adjustment (adaptation). For instance stereotypes are
widely used in information technology systems(Kay, 2000)  and represent a set of
default attributes that often co‐occur in users or in a certain group of people.
The different stereotypes that have been build differ in granularity of detail
and complexity.


\subsection{Adaptive Educational Games.} 

Adaptive Educational Games (AEGs) are complicated educational games that combine
ideas from several investigations areas, to increase the student’s learning
experience (Peeters, et al., 2012a). These are especially based on  serious
games: computer games with an educational approach, where things are taught to
students by using a  playful idea ( Korteling, et al., 2011; Johnson, et al.,
2005). For instance an AEG is a training application  for fire fighters, letting
the fire fighters train their skills and knowledge in a safe on a virtual
environment.
  
The objective of the user model in an AEGs is to optimize the learning process
and outcome.   The user information is considered with the advance and knowledge
of the student, but also with the  student’s mental and cognitive
characteristics. The gained data can be used to adjust the content,
presentation, and system behavior to the  student’s need, for example, by
adjusting the content, tone, or amount of presented feedback. Adaptive computer
games have a lot of processing power available for personalization,  making a
complex and interesting domain for user modeling.

\subsection{Methods for user modeling.} 

In the user modeling topic, researchers have proposed more general design
methods and frameworks to guide the developers in the process of user modeling.
These general methods are useful in research projects, where the knowledge can
be reused to adjust the user model to the system’s characteristics. Also in
commercial applications,  these general methods have proven to be useful (Brun
et al., 2010), because they make it easier and more feasible to implement
personalization into a system. In early work, the process of user modeling was
mostly based on the intuition and experience of the developer or researcher. In
recent work, the techniques of user modeling were essentially based on the
intuition and expertise of the developer or researcher. As the user modeling
research field evolved, there has been put much effort in creating a general way
for designing and constructing a user model, by basing decisions on more
empirical grounds and by defining methods applicable to the whole field (Kobsa,
2001; Durrani, 1997).

Frameworks,  methodologies, and architectures have been developed, defining the
strict process, restrictions and choices on how to design and build a user
model.  In the early days of user modeling, the focus was put on developing one
method applicable to the user modeling field as a whole. However, user modeling
is a very cross‐disciplinary research subject. Therefore, throughout the
decades, the user modeling area of research has been influenced by the important
research topics and trends of their time. For example, when information
technologies became a major subject in the early nineties, user modeling methods
were also mostly focused on the application of stereotypes, knowledge bases, and
logic to define a user model. With the rise of the Internet, the objectives of
the user modeling field change to Web-oriented applications and all the specific
problems that arise with this. Thus this connection,  also the general user
modeling methods that were developed, were focused on the popular research
domains of their time (Kay et al., 2012). The main approaches to user modeling
did not change, but the specific filling‐in of the user model, such as which
technology to apply, did change. In this sense development of user modeling as a
whole, most researchers eventually agreed that one method to solve all problems
is not possible (McTear, 1993; Kobsa, 2001). Instead, a broad range of  generic
user modeling methods has been developed (Fischer, 2001); each of which supports
only a few of the very different manifestations of personalization. In the rest
of this section, the general user modeling architecture and the most interesting
general and domain specific methods will be shortly discussed.







\section{Interactive Evolutionary Computation.}
\begin{figure*}
\captionsetup{justification=centering,margin=2cm}
\centering
\setlength\fboxsep{0pt}
\setlength\fboxrule{0.7pt}
\fbox{\includegraphics[width=10cm,height=10cm,keepaspectratio]{img/restaurant-model.png}}
\caption{User interface of the restaurant model.}
\label{fig:restaurantmodel}       
\end{figure*}


\section{Fuzzy Logic.}

\begin{equation}\label{eq:prediction}
\displaystyle \mu_A\cap \mu_B = T(\mu_A(x),\mu_ B(x))  
\end{equation}

\section{Gamification.}



\chapter{Proposed user-centered framework}

The fundamental goal of this research is to develop a user-centered framework
for interactive evolutionary computation (IEC) in order to increase users\'
participation and also to minimize the amount of evaluations needed for the
evolutionary process in given Web-based IEC application.

In this chapter an explanation of the proposed framework is described. The
different techniques used, such as user modeling, fuzzy logic, and human-
interaction.

This framework is presented in figure \ref{fig:uc_framework}. Each of the
entities of this framework will be explained in detail in following sections.

\begin{figure*}
	\captionsetup{justification=centering,margin=2cm}
	\centering
	\setlength\fboxsep{0pt}
	\setlength\fboxrule{0.7pt}
	\fbox{\includegraphics[width=10cm,height=10cm,keepaspectratio]{img/framework.png}}
	\caption{User-Centerd Framework.}
	\label{fig:uc_framework}
\end{figure*}

\section{Users}

Users are a central part of this proposed framework as it aims to study their
behavior when interacting 
with individuals or phenotypes within interactive evolutionary
algorithms and other tasks that may exist within IEC systems. Thus users in this
proposal are entities interacting with an evolutionary computation that has the
fundamental purpose of evaluating individuals of a population replacing the
fitness function according to their preferences and mood, among others
\cite{takagi1998interactive}. This form of evaluate individual is given a
subjectively way.

On the other hand in order to capture the attention of users and possibly
increase their participation, currently there exist some basic actions that
users can perform in a given system, going from to be able to access a system
through login mechanism, once the user has logged-in in to the system the users
can interact with different actions, for example the inviting action is when
users can be invited through social networks or maybe from person to person.
Another example is the sharing action that can occur when users want to share
something of their interest through their social networks. In this sense the
action of posting is when users want to put something on their wall. 
Likewise the storing action occurs
when users want to retain permanently information that are of their interest.
Finally the browsing action occurs when users are exploring content for their
needs. All these actions go beyond only evaluating individuals.

\begin{figure*}
	\captionsetup{justification=centering,margin=2cm}
	\centering
	\setlength\fboxsep{0pt}
	\setlength\fboxrule{0.7pt}
	\fbox{\includegraphics[width=10cm,height=10cm,keepaspectratio]{img/users.png}}
	\caption{Users actions on interactive evolutionary systems.}
	\label{fig:users}
\end{figure*}

In order to start the task of evaluating individuals is necessary that users
access the system through a login mechanism. This mechanism consists of
providing a username and a password in order to grant access to the system. All
users accessing in this way they are considered active users within an IEC
system.

For the evaluating action is proposed that users evaluate individuals
indirectly, this means users can evaluate accepting indirect recommendations of
friends that are currently active in the system. These recommendations can be
store individuals in the system of users who know each other.

Also the browsing action is proposed within IEC systems, this means that users
can explore information that other active users in the system are generating.

Finally the participation activity is proposed. This can be divided into four
different actions as follows:


\begin{itemize}
\item The Inviting action occurs when a user invites another to the system through social networks or from person to person.
\item The sharing action occurs when users share their individual creations with others active users in the system.
\item The posting action occurs when published what they are doing within the system.
\item The storing action occurs when users keep individuals in a collection, the collections concept will be discussed later in this chapter.
\end{itemize}

All the mention above is shown in figure \ref{fig:users}

\section{Individual}

Individuals or phenotypes are entities that form the population for the evolutionary
algorithm. In particular for this work the genotype of individuals 
are animated digital
paintings, every individual's phenotype in the population is defined by a chromosome.
This chromosome defines the behavior of the individual, so that it consists of a
vector of real numbers of fifteen genes, where each gen define a particular
behavior in the painting. This individuals combined each other to generated new
individual in the population following the genetic operators such as selection,
crossover and mutation. An example of a individual in this case it is
illustrated in figure \ref{fig:individual}



\begin{figure*}
\captionsetup{justification=centering,margin=2cm}
\centering
\setlength\fboxsep{0pt}
\setlength\fboxrule{0.7pt}
\fbox{\includegraphics[width=10cm,height=10cm,keepaspectratio]{img/individual.png}}
\caption{Individual representation.}
\label{fig:graph}
\end{figure*}


\section{User Model}
Now that both, users an individuals have been defined we can
model the users behavior have with respect to individuals in a graph-based user
model.

The reason for using a graph to model the user-individual is because it
can express in a simple way the behavior of these two entities.  
for example a vertex user and individual vertex connected through a edge 
representing the semantics of the relationship 
as Figure \ref{fig:User-Individual} shows. Each of these vertices contain
properties as well as the edges do; these properties will be explained as follows.

\begin{figure*}
\captionsetup{justification=centering,margin=2cm}
\centering
\setlength\fboxsep{0pt}
\setlength\fboxrule{0.7pt}
\fbox{\includegraphics[width=5cm,height=5cm,keepaspectratio]{img/user_individual.png}}
\caption{User-Individual.}
\label{fig:User-Individual}
\end{figure*}

In this proposed graph-based user-individual model the vertices are represented 
by a set of vertices or nodes of USERS, INDIVIDUALS, and COLLECTIONS. 
The edges are defined by a set of relationships as LIKES, 
KNOWS, PARENT, HAS that represents the relationships between the vertices 
as seen in figure \ref{fig:User-Individual}.

%Where was this explained?
We already explained the meaning of USERS and INDIVIDUALS, now an explanation is
given for the concept of COLLECTIONS.

In this proposal, a collection is defined as a container where users can 
store selected individuals (paintings) laid within the collection. A collection 
is created when the user wants to permanently store individuals.

Now the edges in this proposal are explained:

\begin{itemize}
\item The  LIKES edge represents a preference relationship that 
a user has with respect to an individual.
\item The  PARENT edge represents an ancestor of an individual,
 in other words where the individual comes from.
\item Likewise the HAS edge represents that certain user has a collection.
\item Finally the KNOWS edge represents a relationship of friendship between users
and is established by their social network.
\end{itemize}

The above mentioned is illustrated by Figure~\ref{fig:Nodes_Edges}.

\begin{figure*}
\captionsetup{justification=centering,margin=2cm}
\centering
\setlength\fboxsep{0pt}
\setlength\fboxrule{0.7pt}
\fbox{\includegraphics[width=10cm,height=10cm,keepaspectratio]{img/user_individual_collections.png}}
\caption{Nodes And Edges Representation.}
\label{fig:Nodes_Edges}
\end{figure*}


In each vertex is necessary to store knowledge about users, individuals and
also for the collections, this is known as a property.

It is worth to mention that all vertices have a property called “element\_type”
among others. This property is used to identify which type of vertex is, for
example if vertex corresponds to a user then this property labels the vertex as a
“person” type. Likewise if the vertex corresponds to an individual this property
labels it as an “individual” and is the same for any vertex we want to add to the
model. This property also has the purpose of being able to do operations on the
graph according to the vertex type.

The properties for the vertex “u” are the following:

\begin{itemize}
\item id.
\item created.
\item name.
\item element\_type.
\end{itemize}

The “id” property defines that the vertex u has a unique identifier in the set
of vertices, which means that there will not be duplicate vertices. The property
“created” defines the creation date of the vertex $u$. The property “name”
defines the name of the vertex u. The property “element\_type” defines the
element type that will be the vertex $u$.

In figure \ref{fig:User_node} the vertex $u$ properties are shown.

\begin{figure*}
\captionsetup{justification=centering,margin=2cm}
\centering
\setlength\fboxsep{0pt}
\setlength\fboxrule{0.7pt}
\fbox{\includegraphics[width=5cm,height=5cm,keepaspectratio]{img/user_node.png}}
\caption{User Properties.}
\label{fig:User_node}
\end{figure*}

The properties for the vertex $i$ are the following:
\begin{itemize}
\item id.
\item created.
\item element\_type.
\item chromosome.
\item views.
\end{itemize}

The “id” property defines that the vertex $i$ has a unique identifier in the set
of vertices, which means that there will not be duplicate vertices. The
“chromosome” property defines the chromosome of the individual representing the
vertex $i$. which as defined above in this section. The property “views” defines
the amount that users have seen this vertex $i$ The property “element\_type”
defines the element type that will be the vertex $i$. The property “created”
defines the creation date of the vertex $i$.



In figure \ref{fig:Individual_node} the vertex $i$ properties are shown.

\begin{figure*}
\captionsetup{justification=centering,margin=2cm}
\centering
\setlength\fboxsep{0pt}
\setlength\fboxrule{0.7pt}
\fbox{\includegraphics[width=5cm,height=5cm,keepaspectratio]{img/individual_node.png}}
\caption{Individual Properties.}
\label{fig:Individual_node}
\end{figure*}

The properties for the vertex “c” are the following:

\begin{itemize}
\item id.
\item element\_type.
\item created.
\item name.
\end{itemize}

The “id” property defines that the vertex $c$ has a unique identifier in the set
of vertices, which means that there will not be duplicate vertices. The
property “name” defines the name of the vertex $c$. The property “created”
defines the creation date of the vertex $c$.

In figure \ref{fig:Collection_node} the vertex $c$ properties are shown.

\begin{figure*}
\captionsetup{justification=centering,margin=2cm}
\centering
\setlength\fboxsep{0pt}
\setlength\fboxrule{0.7pt}
\fbox{\includegraphics[width=5cm,height=5cm,keepaspectratio]{img/collection_node.png}}
\caption{Collection Properties.}
\label{fig:Collection_node}
\end{figure*}

In the same way that the vertices have properties also the edges, these
properties are defined as follows.

The “LIKES” edge has the following properties:

\begin{itemize}
\item id.
\item created.
\item rate.
\end{itemize}

The “id” property defines that the “LIKES” edge has a unique identifier in the
set of edges, which means that there will not be duplicate edges. The property
“created” defines the date of creation of the this edge. The property “rate”
defines the rate that the users give to the individual store in this edge.

In figure \ref{fig:Likes_edge} the edge “LIKES” properties and the relationships
with the vertices $u$ and $i$ are shown.


\begin{figure*}
\captionsetup{justification=centering,margin=2cm}
\centering
\setlength\fboxsep{0pt}
\setlength\fboxrule{0.7pt}
\fbox{\includegraphics[width=5cm,height=5cm,keepaspectratio]{img/edge_properties_likes.png}}
\caption{LIKES edge Properties.}
\label{fig:Likes_edge}
\end{figure*}

The “KNOWS” edge has the following properties.
\begin{itemize}
\item id.
\item created.
\end{itemize}

The “id” property defines that the edge “KNOWS” has a unique identifier in the
set of edges, which means that there will not be duplicate edges. The property
“created” defines the date of creation of the edge.

In figure \ref{fig:Knows_edge} the edge “KNOWS” properties and the relationships between vertex “u”
are shown.

\begin{figure*}
\captionsetup{justification=centering,margin=2cm}
\centering
\setlength\fboxsep{0pt}
\setlength\fboxrule{0.7pt}
\fbox{\includegraphics[width=5cm,height=5cm,keepaspectratio]{img/edge_properties_knows.png}}
\caption{KNOWS edge Properties.}
\label{fig:Knows_edge}
\end{figure*}

The “PARENT” edge has the following properties, and represent the parents of a
new individual.
\begin{itemize}
\item id.
\item created.
\end{itemize}

The “id” property defines that the edge “PARENT” has a unique identifier in the
set of edges, which means that there will not be duplicate edges. The property
“created” defines the date of creation of the edge.

In figure \ref{fig:Parent_edge} the edge “PARENT” properties and the
relationship between vertex $i$ are shown.

\begin{figure*}
\captionsetup{justification=centering,margin=2cm}
\centering
\setlength\fboxsep{0pt}
\setlength\fboxrule{0.7pt}
\fbox{\includegraphics[width=5cm,height=5cm,keepaspectratio]{img/edge_properties_parent.png}}
\caption{PARENT edge Properties.}
\label{fig:Parent_edge}
\end{figure*}

The edge “HAS” has the following properties, and represent the parents of a new individual.
\begin{itemize}
\item id.
\item created.
\end{itemize}

The “id” property defines that the edge “HAS” has a unique identifier in the set
of edges, which means that there will not be duplicate edges. The property
“created” defines the date of creation of the edge.

In figure \ref{fig:Has_edge} the edge “HAS” properties and the relationship
between vertices $u$, $i$ and $c$ are shown.

\begin{figure*}
\captionsetup{justification=centering,margin=2cm}
\centering
\setlength\fboxsep{0pt}
\setlength\fboxrule{0.7pt}
\fbox{\includegraphics[width=5cm,height=5cm,keepaspectratio]{img/edge_properties_has.png}}
\caption{HAS edge Properties.}
\label{fig:Has_edge}
\end{figure*}

% \begin{equation*}\label{eq:graphRelDef}
% \displaystyle
% \begin{split}
% V &= \{[u_1,u_2,u_3,...,u_n],[i_1,i_2,i_3,...,i_n],[c_1,c_2,c_3,...,c_n]\},\\
% E&= \{[l_1,l_2,l_3,..,l_n],[p_1,p_2,p_3,...,p_n],[h_1,h_2,h_3,...,h_n],[k_1,k_2,k_3,...,k_n]\}\\
% \end{split}
% \end{equation*}

Figure \ref{fig:user_moder} shows an example of the graph-based
user-individual model.

\begin{figure*}
\captionsetup{justification=centering,margin=2cm}
\centering
\setlength\fboxsep{0pt}
\setlength\fboxrule{0.7pt}
\fbox{\includegraphics[width=8cm,height=8cm,keepaspectratio]{img/model_representation.png}}
\caption{Graph-based user-individual model.}
\label{fig:user_moder}
\end{figure*}

\chapter{Fitness Estination}

This chapter explains a proposed strategy to calculate the fitness of
individuals in web-based IEC applications
using fuzzy logic. Before showing our strategy, it is necessary to explain how
the individual evaluation is made in the EvoDrawing application. Figure
\ref{fig:UI_ED} shows the user interface. 
The main goal of this interaction is the evaluation of individuals, but
the first user action before starting to evaluate, is to login through the
Facebook \cite{facebook} social
platform account. The evaluation takes place through a
five star rating selection; this rate represents the degree of user
preference for an individual. The application keeps record of every user
activity by using the activity stream standard [14]. %%%%% Fix reference - Mario
In these particular case activities represent the user\'s experience.

\begin{figure*}
\captionsetup{justification=centering,margin=2cm}
\centering
\setlength\fboxsep{0pt}
\setlength\fboxrule{0.7pt}
\fbox{\includegraphics[width=12cm,height=10cm,keepaspectratio]{img/UI_ed01.png}}
\caption{User interface ED01.}
\label{fig:UI_ED}
\end{figure*}

In this research we propose the use of fuzzy logic[10] in order to obtain a %% Fix references please - Mario
difuzzify value to be used to calculate the individual fitness trough a fitness
function expression. It is used by modeling a fuzzy Mamdani type inference

system [5] [6] as figure \ref{fig:fis01} shown. This model was designed empirically. %% Fix references please - Mario
The model consists of two input variables, which are the preference and the
experience of the user as well as an output that we called fuzzy rate. The first
one has 3 linguistic variables, which are low, medium and high, representing the 
% These are not variables. Low is a linguistic variable?? 
preference with triangular membership functions over a range of 1 to 5. The
second one also has three linguistic variables, which are low, medium and high
representing the experience with triangular membership functions over a range of
1 to 100. Finally we have the output consisting of three linguistic variables
bad, normal and good representing the fuzzy rate with triangular membership
functions in a range of 1-100.

\begin{figure*}
\captionsetup{justification=centering,margin=2cm}
\centering
\setlength\fboxsep{0pt}
\setlength\fboxrule{0.7pt}
\fbox{\includegraphics[width=12cm,height=10cm,keepaspectratio]{img/fuzzy_system_2_1.png}}
\caption{Fuzzy System Mamdani Type.}
\label{fig:fis01}
\end{figure*}

Below we show the rules IF-THEN of the fuzzy system:

\begin{enumerate}
	\item \textit{If \textbf{preference} is low and
		\textbf{experience} is low then \textbf{fuzzy\_rate} is bad.}
	\item \textit{If \textbf{preference} is mid and
		\textbf{experience} is low  then \textbf{fuzzy\_rate} is bad.}
	\item \textit{If \textbf{preference} is high and
		\textbf{experience} is low  then \textbf{fuzzy\_rate} is normal.}
	\item \textit{If \textbf{preference} is low and
		\textbf{experience} is mid then \textbf{fuzzy\_rate} is bad.}
	\item \textit{If \textbf{preference} is mid and
		\textbf{experience} is mid  then \textbf{fuzzy\_rate} is normal.}
	\item \textit{If \textbf{preference} is high and
		\textbf{experience} is mid  then \textbf{fuzzy\_rate} is good.}
	\item \textit{If \textbf{preference} is low and
		\textbf{experience} is high then \textbf{fuzzy\_rate} is normal.}
	\item \textit{If \textbf{preference} is mid and
		\textbf{experience} is high  then \textbf{fuzzy\_rate} is good.}
	\item \textit{If \textbf{preference} is high and
		\textbf{experience} is high  then \textbf{fuzzy\_rate} is good.}

\end{enumerate}

These rules will give us a fuzzy rate value as a result, this value needs to
be defuzzified by the centroid method in order to be used in our fitness
expression, given by the equation\ref{eq:fitfunc_02}. This expression is responsible
of representing the individual fitness.

\begin{equation}\label{eq:fitfunc_02}
\displaystyle fitness=\frac{\sum_{i=0}^{n}x_{i}+f(y_{i})}{\sum_{i=0}^{n}f(y_{i})}
\end{equation}

Where $n$ represents  the number of  users that have evaluated the
individual, $x$ is the rate of preference for the  individual  given by the user,
$y$ is a function that calls the fuzzy system in order to have the fuzzy rate.
This function has as a parameter the rate $x$ and the user experience level. 
The user experience level
is given  by the  total activities that user has at the moment. In each
activity we assign the score, for example if the user log in (join) to the
application we assign 5 points, if the user evaluates (likes) an individual we
give 3 points, etc; in figure 5 shows the flow for assigning fitness to
the individual. 


\include{caseStudy}
\chapter{Results} \label{sec:6} This chapter presents the results obtained from
the study case discussed in the previous chapter.

\section{EvoDrawing01}

The experiment EvoDrawing01 generated a graph user-model with the following data
that is shown in table \ref{tab:dataGenerated_1}

\begin{table}
\small
\caption{Data generated in graph-based user modeling.}
\label{tab:dataGenerated_1}
\centering
\small
\begin{tabular}{p{3cm} p{3cm} p{3cm} }
\hline\noalign{\smallskip}
  & Data &  \\
\noalign{\smallskip}\hline\noalign{\smallskip}
\small{Nodes} & \small{595} & \\ \hline
\small{Relationships} & \small{2220} & \\ \hline

\noalign{\smallskip}\hline
\end{tabular}
\end{table}


These data is the total of nodes as well as relationships. Also in
table \ref{tab:totalUsers_12} is the total of volunteer active users.

\begin{table}
\small
\caption{Total number of volunteers active users.}
\label{tab:totalUsers_12}
\centering
\small
\begin{tabular}{p{3cm} p{3cm} p{3cm} }
\hline\noalign{\smallskip}
  & Users &  \\
\noalign{\smallskip}\hline\noalign{\smallskip}
\small{Total } & \small{53} & \\ \hline
\noalign{\smallskip}\hline
\end{tabular}
\end{table}

In order to observe which users have better social interconnectivity within the
experiments it was decided to make a relation of number known users that the
user’s have. This relation is shown in table \ref{tab:knownUsers_1}

\begin{table}
\small
\caption{Number of known among users.}
\label{tab:knownUsers_1}
\centering
\small
\begin{tabular}{p{3cm} p{3cm}  }
\hline\noalign{\smallskip}
 User name & Number of known users \\
\noalign{\smallskip}\hline\noalign{\smallskip}
\small{Chriss de Blanc } & \small{7}  \\ \hline
\small{Alejandro Salcido } & \small{4}  \\ \hline
\small{Jonathan Amezcua Aguiluz } & \small{1}  \\ \hline
\small{Daniela Sanchez } & \small{1}  \\ \hline
\small{Jennifer Llamas} & \small{1}  \\ \hline
\small{Iliana Dl } & \small{1}  \\ \hline
\small{Xochilt Ramirez Garcia } & \small{1}  \\ \hline
\small{Manuel Elizondo } & \small{1}  \\ \hline
\small{Julian Torres } & \small{1}  \\ \hline
\small{Data Back } & \small{1}  \\ \hline
\small{Frank Arce } & \small{1}  \\ \hline
\small{Gustavo Vargas } & \small{1}  \\ \hline
\noalign{\smallskip}\hline
\end{tabular}
\end{table}

Likewise in table \ref{tab:knownUsers_1} interconnectivity having a particular
user with other users is shown. The degree of relationship of users is
associated with the number of friends known within the application. This means
that we have the degree of influence among participants. For example  "Chriss
Blanc" user has a degree of relatedness 7 as shown in figure
\ref{fig:guserknown_1} indicating that this particular user can have more
influence on the decisions of others. Moreover the user "Alejandro Salcido" has
the second highest degree of relationship to other users as shown in figure
\ref{fig:guserknown_1}.

\begin{figure*}
%\captionsetup{justification=centering,margin=1cm}
\centering
\fbox{\includegraphics[scale=0.75]{img/users_known_1.PNG}} %[width=0.7\textwidth]
\caption{Graph users with greater interconnectivity.}
\label{fig:guserknown_1}
\end{figure*}

In figure \ref{fig:guserknown_1} shows users more connected in the graph. This
may represent the degree of impact of a user and the possible influence that may
have on the decisions of other users. For example, if the user with the greatest
impact is affected in any decision likely other users may be affected in some
way.


\begin{table}
\small
\caption{Total number of individuals generated.}
\label{tab:totalIndividuals_12}
\centering
\small
\begin{tabular}{p{3cm} p{3cm} p{3cm} }
\hline\noalign{\smallskip}
  & Individuals &  \\
\noalign{\smallskip}\hline\noalign{\smallskip}
\small{Total } & \small{500} & \\ \hline
\noalign{\smallskip}\hline
\end{tabular}
\end{table}


Table \ref{tab:totalIndividuals_12} presents the total number of individuals
generated within the graph.

\begin{table}
\small
\caption{Sample of 30 individuals evaluated from.}
\label{tab:totalIndividuals_1}
\centering
\small
\begin{tabular}{p{3cm} p{4cm} p{3cm} p{3cm}}
\hline\noalign{\smallskip}
 id & Chromosome & Views & Likes  \\
\noalign{\smallskip}\hline\noalign{\smallskip}
\small{pop:individual:107} & \small{[125, 30, 0, 1, 0, 0, 3, 0, 1, 0, 0, 0, 0, 2, 1]}
& \small{20} & \small{20}\\ \hline
\small{pop:individual:133} & \small{[143, 15, 1, 1, 0, 1, 3, 0, 1, 0, 2, 0, 0, 1, 1]}
& \small{15} & \small{15}\\ \hline
\small{pop:individual:109} & \small{[125, 30, 1, 1, 0, 0, 3, 0, 1, 0, 0, 0, 0, 0, 1]}
& \small{15} & \small{15}\\ \hline
\small{pop:individual:37} & \small{[87, 64, 1, 1, 1, 1, 4, 0, 1, 0, 0, 0, 1, 2, 3]}
& \small{14} & \small{14}\\ \hline
\small{pop:individual:36} & \small{[95, 71, 0, 1, 1, 0, 3, 1, 0, 0, 0, 1, 1, 1, 2]}
& \small{14} & \small{14}\\ \hline
\small{pop:individual:215} & \small{[138, 29, 1, 1, 0, 1, 3, 0, 1, 1, 1, 0, 0, 1, 1]}
& \small{13} & \small{13}\\ \hline
\small{pop:individual:228} & \small{[42, 58, 0, 0, 1, 1, 4, 0, 0, 0, 2, 0, 0, 0, 3]}
& \small{12} & \small{12}\\ \hline
\small{pop:individual:48} & \small{[51, 73, 0, 0, 0, 0, 2, 0, 0, 1, 3, 0, 1, 0, 3]}
& \small{12} & \small{12}\\ \hline
\small{pop:individual:39} & \small{[60, 12, 1, 1, 1, 1, 4, 1, 1, 1, 0, 1, 0, 1, 1]}
& \small{12} & \small{12}\\ \hline
\small{pop:individual:94} & \small{[49, 71, 0, 0, 1, 0, 4, 1, 0, 0, 1, 1, 1, 0, 2]}
& \small{11} & \small{11}\\ \hline
\small{pop:individual:194} & \small{[87, 64, 0, 1, 1, 1, 1, 3, 0, 0, 0, 0, 1, 2, 3]}
& \small{11} & \small{11}\\ \hline
\small{pop:individual:75} & \small{[97, 66, 0, 0, 1, 1, 3, 0, 1, 0, 3, 0, 1, 0, 1]}
& \small{11} & \small{11}\\ \hline
\small{pop:individual:105} & \small{[125, 30, 0, 1, 0, 0, 3, 0, 1, 0, 0, 1, 0, 0, 1]}
& \small{11} & \small{11}\\ \hline
\small{pop:individual:306} & \small{[87, 64, 1, 1, 1, 1, 4, 1, 1, 1, 1, 0, 1, 2, 3]}
& \small{13} & \small{10}\\ \hline
\small{pop:individual:326} & \small{[138, 29, 1, 1, 0, 1, 3, 0, 1, 1, 1, 0, 0, 0, 0]}
& \small{10} & \small{10}\\ \hline
\small{pop:individual:82} & \small{[81, 8, 1, 0, 0, 1, 4, 1, 1, 1, 2, 0, 0, 2, 1]}
& \small{10} & \small{10}\\ \hline
\small{pop:individual:252} & \small{[125, 30, 0, 1, 0, 0, 3, 0, 1, 1, 3, 0, 1, 0, 3]}
& \small{10} & \small{10}\\ \hline
\small{pop:individual:280} & \small{[53, 63, 1, 1, 0, 1, 3, 0, 1, 1, 0, 0, 0, 2, 1]}
& \small{9} & \small{9}\\ \hline
\small{pop:individual:178} & \small{[53, 63, 1, 1, 0, 1, 3, 0, 1, 1, 0, 0, 0, 2, 1]}
& \small{9} & \small{9}\\ \hline
\small{pop:individual:108} & \small{[42, 58, 0, 0, 1, 1, 3, 0, 1, 0, 1, 1, 0, 0, 1]}
& \small{10} & \small{9}\\ \hline
\small{pop:individual:231} & \small{[125, 30, 0, 1, 0, 0, 3, 0, 1, 1, 3, 0, 1, 0, 3]}
& \small{0} & \small{9}\\ \hline
\small{pop:individual:147} & \small{[122, 38, 1, 0, 0, 0, 0, 3, 0, 1, 0, 0, 0, 0, 0]}
& \small{9} & \small{9}\\ \hline
\small{pop:individual:181} & \small{[143, 15, 1, 1, 0, 1, 3, 0, 1, 0, 0, 0, 0, 0, 1]}
& \small{9} & \small{9}\\ \hline
\small{pop:individual:151} & \small{[49, 27, 1, 0, 0, 0, 3, 0, 0, 1, 2, 0, 1, 0, 3]}
& \small{9} & \small{9}\\ \hline
\small{pop:individual:34} & \small{[125, 30, 0, 1, 0, 0, 3, 0, 1, 0, 0, 0, 0, 2, 1]}
& \small{8} & \small{8}\\ \hline
\small{pop:individual:185} & \small{[122, 38, 0, 0, 0, 1, 1, 1, 1, 1, 3, 0, 1, 1, 1]}
& \small{8} & \small{8}\\ \hline
\small{pop:individual:88} & \small{[49, 62, 0, 1, 1, 1, 1, 1, 1, 1, 0, 0, 0, 2, 2]}
& \small{8} & \small{8}\\ \hline
\small{pop:individual:176} & \small{[125, 30, 0, 1, 0, 0, 3, 0, 1, 1, 3, 0, 1, 0, 3]}
& \small{8} & \small{8}\\ \hline
\small{pop:individual:234} & \small{[42, 58, 0, 0, 1, 1, 3, 0, 1, 0, 1, 0, 0, 0, 1]}
& \small{8} & \small{8}\\ \hline
\noalign{\smallskip}\hline
\end{tabular}
\end{table}

Table \ref{tab:totalIndividuals_1} contains a sample of 30/500 individuals were
generated in the experiment. Where we present the unique identifier of the
individual, its chromosome, as well as the number of views, likes available to
the individual. This results are useful to observe individuals have been better
evaluated by users.

\begin{table}
\small
\caption{Level of user participation.}
\label{tab:userParticipation_1}
\centering
\small
\begin{tabular}{p{4cm} p{4cm}}
\hline\noalign{\smallskip}
 User name & Participation   \\
\noalign{\smallskip}\hline\noalign{\smallskip}
\small{Ana Laura Lopez} & \small{116} \\ \hline
\small{Mario García Valdez} & \small{100} \\ \hline
\small{Chriss de Blanc} & \small{93} \\ \hline
\small{Xochilt Ramirez Garcia} & \small{85} \\ \hline
\small{Carlos David Gallardo Pérez} & \small{73} \\ \hline
\small{Ulises Reus} & \small{70} \\ \hline
\small{Aaron Gutierrez Urbina} & \small{58} \\ \hline
\small{Cesar López} & \small{49} \\ \hline
\small{Hector Beltran Medrano} & \small{48} \\ \hline
\small{Luis Alfonso Felix Garcia} & \small{45} \\ \hline
\small{Data Back} & \small{39} \\ \hline
\small{Amaury Hernandez Aguila} & \small{32} \\ \hline
\small{Osmar Herrera Duran} & \small{31} \\ \hline
\small{Jorman Gtz} & \small{29} \\ \hline
\small{Alexis Campos Lopez} & \small{29} \\ \hline
\small{Melissa Muñoz Montes} & \small{28} \\ \hline
\small{David Gallegos} & \small{23} \\ \hline
\small{Jose Carlos} & \small{21} \\ \hline
\small{Tomás Perrín} & \small{21} \\ \hline
\small{Manuel Elizondo} & \small{20} \\ \hline


\noalign{\smallskip}\hline
\end{tabular}
\end{table}




Table \ref{tab:userParticipation_1} shows the results of the level of user
participation in the experiment. These were obtained by counting the vicinity of
nearest nodes from the base node in this case each user node.



\begin{figure*}
%\captionsetup{justification=centering,margin=1cm}
\centering
\fbox{\includegraphics[scale=0.75]{img/Visual_represntation_1.PNG}} %[width=0.7\textwidth]
\caption{Visual representation of user participation in EvoDrawing01.}
\label{fig:userP_1}
\end{figure*}

In figure \ref{fig:userP_1} we present a visual representation of user
participation this experiment where the y-axis represents the level of
participation and the x axis represents the number of users who participated in
this experiment.

\begin{figure*}
%\captionsetup{justification=centering,margin=1cm}
\centering
\fbox{\includegraphics[scale=0.75]{img/weibull_1.PNG}} %[width=0.7\textwidth]
\caption{Weibull fit data representation.}
\label{fig:weibull_1}
\end{figure*}


\section {EvoDrawing02}


In the experiment EvoDrawings02 the following data were generated and is presented in table \ref{tab:dataGenerated_2}.


\begin{table}
\small
\caption{Data generated in graph-based user modeling.}
\label{tab:dataGenerated_2}
\centering
\small
\begin{tabular}{p{3cm} p{3cm} p{3cm} }
\hline\noalign{\smallskip}
  & Data &  \\
\noalign{\smallskip}\hline\noalign{\smallskip}
\small{Nodes} & \small{648} & \\ \hline
\small{Relationships} & \small{2596} & \\ \hline

\noalign{\smallskip}\hline
\end{tabular}
\end{table}

These data are generated total of nodes as well as relationships. Also in table
\ref{tab:totalUsers_1} is the total of volunteer active users is presented.

The total number of users who participate voluntarily shown in the table x.

\begin{table}
\small
\caption{Total number of volunteers active users.}
\label{tab:totalUsers_1}
\centering
\small
\begin{tabular}{p{3cm} p{3cm} p{3cm} }
\hline\noalign{\smallskip}
  & Users &  \\
\noalign{\smallskip}\hline\noalign{\smallskip}
\small{Total } & \small{54} & \\ \hline
\noalign{\smallskip}\hline
\end{tabular}
\end{table}

Like the previous experiment which wanted to observe users had better social
interconnectivity within this experiment, it was decided to see this
relationship that correspond to the number of known users. This relationship is
presented in table \ref{tab:totalUsers_1} where the top 10 user  with social
interconnectivity are shown.


\begin{table}
\small
\caption{A sample of the top 10 users interconnected users.}
\label{tab:knownUsers_2}
\centering
\small
\begin{tabular}{p{3cm} p{3cm}  }
\hline\noalign{\smallskip}
 User name & Number of known users \\
\noalign{\smallskip}\hline\noalign{\smallskip}
\small{Rogelio UR} & \small{10}  \\ \hline
\small{Barbara Sandoval} & \small{9}  \\ \hline
\small{Evelyn Macedo} & \small{8}  \\ \hline
\small{Chriss de Blanc} & \small{8}  \\ \hline
\small{Cesar Rojas} & \small{8}  \\ \hline
\small{Jasiel Calzada} & \small{8}  \\ \hline
\small{Tonyy Maldonado} & \small{7}  \\ \hline
\small{Silvano Peraza} & \small{7}  \\ \hline
\small{Hector Beltran Medrano} & \small{6}  \\ \hline
\small{Juan Ferman Lopez} & \small{6}  \\ \hline
\noalign{\smallskip}\hline
\end{tabular}
\end{table}

In figure \ref{fig:bestIndividuals_2} shows users more interconnectivity in the
graph. This may represent the degree of impact of a user and the possible
influence that may have on the decisions of other users.  For example, if the
user with the greatest impact is affected in any decision likely other users may
be affected in some way. Particularly in this experiment the degree of social
interconnectivity among users is becoming more complex as compared to the
previous experiment.

\begin{figure*}
%\captionsetup{justification=centering,margin=1cm}
\centering
\fbox{\includegraphics[scale=0.75]{img/user_known_2.PNG}} %[width=0.7\textwidth]
\caption{Sample of the top 30 individuals evaluated from a total of 556.}
\label{fig:bestIndividuals_2}
\end{figure*}

\begin{table}
\small
\caption{Sample of 30 individuals evaluated from.}
\label{tab:totalIndividuals_2}
\centering
\small
\begin{tabular}{p{3cm} p{4cm} p{3cm} p{3cm}}
\hline\noalign{\smallskip}
 id & Chromosome & Views & Likes  \\
\noalign{\smallskip}\hline\noalign{\smallskip}
\small{pop:individual:55} & \small{[63, 58, 0, 1, 1, 1, 4, 0, 0, 1, 0, 0, 0, 0, 3]}
& \small{18} & \small{18}\\ \hline
\small{pop:individual:329} & \small{[98, 37, 0, 1, 1, 0, 4, 0, 0, 1, 0, 0, 0, 0, 2]}
& \small{17} & \small{17}\\ \hline
\small{pop:individual:304} & \small{[63, 58, 0, 1, 1, 1, 4, 0, 0, 0, 3, 0, 0, 2, 2]}
& \small{16} & \small{16}\\ \hline
\small{pop:individual:202} & \small{[107, 79, 1, 0, 1, 1, 3, 0, 0, 1, 3, 0, 1, 1]}
& \small{15} & \small{15}\\ \hline
\small{pop:individual:58} & \small{[150, 79, 1, 0, 1, 1, 3, 0, 0, 1, 3, 0, 1, 1, 1]}
& \small{14} & \small{14}\\ \hline
\small{pop:individual:310} & \small{[63, 58, 0, 1, 1, 1, 4, 0, 0, 1, 0, 0, 1, 2, 2]}
& \small{13} & \small{13}\\ \hline
\small{pop:individual:67} & \small{[65, 51, 1, 1, 0, 0, 3, 1, 0, 0, 3, 1, 0, 2, 3]}
& \small{12} & \small{12}\\ \hline
\small{pop:individual:48} & \small{[51, 73, 0, 0, 0, 0, 2, 0, 0, 1, 3, 0, 1, 0, 3]}
& \small{12} & \small{12}\\ \hline
\small{pop:individual:179} & \small{[98, 37, 0, 1, 1, 1, 4, 0, 0, 1, 0, 0, 0, 0, 3]}
& \small{12} & \small{12}\\ \hline
\small{pop:individual:114} & \small{[63, 58, 0, 1, 1, 1, 4, 0, 0, 1, 0, 0, 0, 0, 2]}
& \small{12} & \small{12}\\ \hline
\small{pop:individual:123} & \small{[124, 42, 0, 1, 1, 1, 4, 1, 0, 1, 1, 0, 0, 2, 1]}
& \small{12} & \small{12}\\ \hline
\small{pop:individual:344} & \small{[97, 66, 0, 0, 1, 1, 3, 0, 1, 0, 3, 0, 1, 0, 1]}
& \small{11} & \small{11}\\ \hline
\small{pop:individual:105} & \small{[63, 58, 0, 1, 1, 1, 4, 0, 0, 1, 0, 0, 0, 0, 2]}
& \small{11} & \small{11}\\ \hline
\small{pop:individual:435} & \small{[98, 37, 0, 1, 1, 0, 1, 1, 4, 1, 0, 0, 0, 0, 2]}
& \small{11} & \small{11}\\ \hline
\small{pop:individual:140} & \small{[150, 79, 1, 0, 1, 1, 3, 1, 1, 0, 1, 1, 1, 2, 3]}
& \small{11} & \small{11}\\ \hline
\small{pop:individual:216} & \small{[124, 42, 0, 1, 0, 0, 3, 0, 0, 1, 1, 0, 0, 1, 1]}
& \small{11} & \small{11}\\ \hline
\small{pop:individual:290} & \small{[150, 79, 1, 0, 1, 1, 3, 0, 0, 1, 3, 0, 1, 0, 1]}
& \small{10} & \small{10}\\ \hline
\small{pop:individual:255} & \small{[150, 79, 1, 0, 1, 1, 3, 1, 1, 1, 0, 0, 1, 2, 2]}
& \small{10} & \small{10}\\ \hline
\small{pop:individual:14} & \small{[89, 66, 0, 0, 1, 1, 3, 0, 0, 0, 2, 0, 0, 2, 1]}
& \small{10} & \small{10}\\ \hline
\small{pop:individual:366} & \small{[128, 66, 0, 1, 1, 4, 1, 0, 1, 0, 0, 0, 1, 2, 2]}
& \small{10} & \small{10}\\ \hline
\small{pop:individual:215} & \small{[54, 72, 0, 1, 1, 1, 4, 0, 0, 1, 0, 0, 0, 0, 2]}
& \small{9} & \small{9}\\ \hline
\small{pop:individual:411} & \small{[113, 58, 0, 1, 1, 1, 4, 0, 0, 1, 0, 0, 1, 2, 2]}
& \small{9} & \small{9}\\ \hline
\small{pop:individual:254} & \small{[113, 58, 0, 1, 1, 4, 1, 0, 4, 1, 1, 0, 0, 3]}
& \small{11} & \small{9}\\ \hline
\small{pop:individual:285} & \small{[54, 58, 0, 0, 0, 1, 1, 4, 1, 1, 0, 0, 1, 2, 2]}
& \small{8} & \small{8}\\ \hline
\small{pop:individual:500} & \small{[63, 58, 1, 1, 1, 0, 3, 0, 1, 1, 0, 0, 1, 2, 0]}
& \small{8} & \small{8}\\ \hline
\small{pop:individual:415} & \small{[54, 72, 0, 1, 1, 1, 4, 1, 4, 1, 3, 0, 1, 1]}
& \small{8} & \small{8}\\ \hline
\small{pop:individual:300} & \small{[54, 58, 1, 0, 1, 1, 1, 4, 1, 1, 0, 0, 1, 2, 2]}
& \small{8} & \small{8}\\ \hline
\small{pop:individual:32} & \small{[113, 29, 0, 0, 0, 0, 4, 1, 0, 1, 0, 1, 1, 1, 3]}
& \small{8} & \small{8}\\ \hline
\small{pop:individual:237} & \small{[63, 58, 1, 1, 0, 1, 1, 4, 1, 0, 0, 0, 1, 2, 3]}
& \small{8} & \small{8}\\ \hline
\small{pop:individual:268} & \small{[98, 37, 0, 1, 1, 1, 4, 0, 0, 1, 0, 0, 0, 0]}
& \small{8} & \small{8}\\ \hline
\noalign{\smallskip}\hline
\end{tabular}
\end{table}

Table \ref{tab:totalIndividuals_2} contains a sample of individuals 30/556
generated in the experiment. This table shows the unique identifier of the
individual, its chromosome, as well as the number of views, likes available to
the individual.This results are useful to observe individuals have been better
evaluated by users.

\begin{table}
\small
\caption{Level of user participation.}
\label{tab:userParticipation_2}
\centering
\small
\begin{tabular}{p{4cm} p{4cm}}
\hline\noalign{\smallskip}
 User name & Participation   \\
\noalign{\smallskip}\hline\noalign{\smallskip}
\small{1122212314475816} & \small{122} \\ \hline
\small{1107674982600275} & \small{91} \\ \hline
\small{10207544086753085} & \small{86} \\ \hline
\small{1128990193799346} & \small{85} \\ \hline
\small{1067084180030552} & \small{83} \\ \hline
\small{985591718197586} & \small{77} \\ \hline
\small{969507913124553} & \small{72} \\ \hline
\small{10207004677610003} & \small{63} \\ \hline
\small{1223229694371825} & \small{53} \\ \hline
\small{10153904046011462} & \small{47} \\ \hline
\small{1032494570125948} & \small{44} \\ \hline
\small{995610090523549} & \small{42} \\ \hline
\small{975038365907627} & \small{35} \\ \hline
\small{10207487295119454} & \small{34} \\ \hline
\small{1041989022528624} & \small{32} \\ \hline
\small{471974623003503} & \small{32} \\ \hline
\small{1275844349097672} & \small{31} \\ \hline
\small{978744228875903} & \small{31} \\ \hline
\small{10205734318020434} & \small{31} \\ \hline
\small{10209454397419860} & \small{29} \\ \hline


\noalign{\smallskip}\hline
\end{tabular}
\end{table}

In table \ref{tab:userParticipation_2} shows the results of the level of user
participation in the experiment.These were obtained by counting the vicinity of
nearest nodes from the base node in this case each user node.

\begin{figure*}
%\captionsetup{justification=centering,margin=1cm}
\centering
\fbox{\includegraphics[scale=0.75]{img/visual_representation_2.PNG}} %[width=0.7\textwidth]
\caption{Visual representation of user participation in EvoDrawing01.}
\label{fig:userP_2}
\end{figure*}


In figure \ref{fig:userP_2} we present a visual representation of user
participation of this experiment where the y-axis represents the level of
participation and the x axis represents the number of users who participated in
this experiment.

\begin{figure*}
%\captionsetup{justification=centering,margin=1cm}
\centering
\fbox{\includegraphics[scale=0.75]{img/weibull_2.PNG}} %[width=0.7\textwidth]
\caption{Weibull fit data representation.}
\label{fig:weibull_2}
\end{figure*}




\section{EvoDrawing03} In the experiment EvoDrawings03 the following data were
generated presented in \ref{tab:dataGenerated_3}.

\begin{table}
\small
\caption{Data generated in graph-based user modeling.}
\label{tab:dataGenerated_3}
\centering
\small
\begin{tabular}{p{3cm} p{3cm} p{3cm} }
\hline\noalign{\smallskip}
  & Data &  \\
\noalign{\smallskip}\hline\noalign{\smallskip}
\small{Nodes} & \small{3746 } & \\ \hline
\small{Relationships} & \small{17207 } & \\ \hline

\noalign{\smallskip}\hline
\end{tabular}
\end{table}

The total number of users who participate voluntarily shown in \ref{tab:totalUsers_3}.

\begin{table}
\small
\caption{Total number of volunteers active users.}
\label{tab:totalUsers_3}
\centering
\small
\begin{tabular}{p{3cm} p{3cm} p{3cm} }
\hline\noalign{\smallskip}
  & Users &  \\
\noalign{\smallskip}\hline\noalign{\smallskip}
\small{Total } & \small{68} & \\ \hline
\noalign{\smallskip}\hline
\end{tabular}
\end{table}

In the same way as the previous experiment which wanted to observe users had
better social interconnectivity within this experiment, it was decided a known
ratio of the number of users that have. This relationship is presented in Table
\ref{tab:knownUsers_3} 13 where we show the 10 users with better social
interconnectivity.

\begin{table}
\small
\caption{A sample of the top 10 users level influence on users.}
\label{tab:knownUsers_3}
\centering
\small
\begin{tabular}{p{3cm} p{3cm}  }
\hline\noalign{\smallskip}
 User name & Number of known users \\
\noalign{\smallskip}\hline\noalign{\smallskip}
\small{David Astorga Viramontes} & \small{23}  \\ \hline
\small{Evelyn Macedo} & \small{18}  \\ \hline
\small{Reyes Zuniga} & \small{17}  \\ \hline
\small{Barbara Sandoval} & \small{16}  \\ \hline
\small{Chriss de Blanc} & \small{15}  \\ \hline
\small{Luis Alfonso Felix Garcia} & \small{15}  \\ \hline
\small{Paulina NG} & \small{15}  \\ \hline
\small{Marco Antonio Fuentes Alvarez} & \small{15}  \\ \hline
\small{Jasiel Calzada} & \small{14}  \\ \hline
\small{Hector Beltran Medrano} & \small{13}  \\ \hline
\noalign{\smallskip}\hline
\end{tabular}
\end{table}


\begin{figure*}
%\captionsetup{justification=centering,margin=1cm}
\centering
\fbox{\includegraphics[scale=0.75]{img/user_known_3.PNG}} %[width=0.7\textwidth]
\caption{Social interconnectivity in graph-based user meodel.}
\label{fig:graphKnown_3}
\end{figure*}

In figure \ref{fig:graphKnown_3} shows users more connected in the graph. This
may represent the degree of impact of a user and the possible influence that may
have on the decisions of other users. For example, if the user with the greatest
impact is affected in any decision likely other users may be affected in some
way. Particularly in this experiment the social graph interconnectivity among
users is becoming more complex as compared to the previous experiment as we can
observe that more closely resembles a social network.


\begin{table}
\small
\caption{Total number of individuals.}
\label{tab:totalIndividuals_33}
\centering
\small
\begin{tabular}{p{3cm} p{3cm} p{3cm} }
\hline\noalign{\smallskip}
  & Individuals &  \\
\noalign{\smallskip}\hline\noalign{\smallskip}
\small{Total } & \small{3594} & \\ \hline
\noalign{\smallskip}\hline
\end{tabular}
\end{table}

In table \ref{tab:totalIndividuals_33} shows the total number of individuals
evaluated by users in this experiment.

\begin{table}
\small
\caption{Sample of 30 individuals evaluated from.}
\label{tab:totalIndividuals_32}
\centering
\small
\begin{tabular}{p{3cm} p{4cm} p{3cm} p{3cm}}
\hline\noalign{\smallskip}
 id & Chromosome & Views & Likes  \\
\noalign{\smallskip}\hline\noalign{\smallskip}
\small{pop:individual:3570} & \small{[113, 44, 0, 1, 0, 0, 1, 0, 1, 0, 3, 1, 1, 0, 1]}
& \small{48} & \small{48}\\ \hline
\small{pop:individual:3544} & \small{[150, 44, 0, 0, 1, 1, 1, 0, 0, 1, 1, 0, 1, 0, 0]}
& \small{32} & \small{26}\\ \hline
\small{pop:individual:210} & \small{[63, 58, 0, 1, 1, 1, 4, 0, 0, 0, 3, 0, 0, 2, 2]}
& \small{16} & \small{16}\\ \hline
\small{pop:individual:202} & \small{[113, 5, 0, 1, 1, 0, 4, 0, 1, 1, 2, 1, 0, 2, 2]}
& \small{22} & \small{22}\\ \hline
\small{pop:individual:58} & \small{[61, 70, 0, 0, 1, 0, 1, 0, 0, 0, 2, 0, 1, 0, 3]}
& \small{23} & \small{21}\\ \hline
\small{pop:individual:858} & \small{[113, 1, 1, 1, 1, 0, 1, 1, 1, 3, 0, 1, 1, 0]}
& \small{21} & \small{21}\\ \hline
\small{pop:individual:17} & \small{[113, 69, 0, 1, 1, 1, 3, 0, 1, 1, 0, 0, 1, 0, 3]}
& \small{23} & \small{21}\\ \hline
\small{pop:individual:3636} & \small{[118, 44, 0, 0, 1, 1, 1, 0, 0, 1, 1, 1, 1, 0, 0]}
& \small{22} & \small{21}\\ \hline
\small{pop:individual:33} & \small{[129, 79, 0, 1, 1, 0, 1, 1, 0, 0, 0, 1, 0, 0, 1]}
& \small{20} & \small{20}\\ \hline
\small{pop:individual:74} & \small{[150, 50, 1, 1, 1, 1, 1, 1, 0, 1, 3, 1, 0, 1, 3]}
& \small{21} & \small{19}\\ \hline
\small{pop:individual:65} & \small{[94, 62, 1, 0, 0, 0, 2, 0, 1, 0, 3, 0, 1, 2, 2]}
& \small{20} & \small{19}\\ \hline
\small{pop:individual:50} & \small{[50, 70, 0, 1, 1, 1, 3, 1, 1, 1, 1, 1, 1, 1, 3]}
& \small{19} & \small{19}\\ \hline
\small{pop:individual:52} & \small{[131, 63, 1, 1, 1, 0, 4, 0, 1, 1, 0, 0, 1, 0, 3]}
& \small{20} & \small{19}\\ \hline
\small{pop:individual:2449} & \small{[113, 44, 0, 1, 1, 1, 1, 0, 1, 1, 0, 0, 1, 1, 2]}
& \small{18} & \small{18}\\ \hline
\small{pop:individual:44} & \small{[117, 54, 0, 0, 0, 0, 2, 0, 1, 1, 3, 1, 0, 2, 3]}
& \small{19} & \small{18}\\ \hline
\small{pop:individual:113} & \small{[77, 21, 1, 0, 1, 0, 4, 0, 1, 0, 0, 1, 0, 3]}
& \small{17} & \small{17}\\ \hline
\small{pop:individual:84} & \small{[102, 69, 1, 0, 1, 0, 3, 0, 0, 0, 0, 0, 1, 0, 3]}
& \small{17} & \small{17}\\ \hline
\small{pop:individual:95} & \small{[68, 43, 0, 1, 0, 1, 3, 1, 0, 0, 3, 0, 0, 1, 2]}
& \small{19} & \small{16}\\ \hline
\small{pop:individual:22} & \small{[137, 20, 0, 0, 0, 0, 0, 1, 0, 0, 3, 0, 1, 1, 1]}
& \small{17} & \small{16}\\ \hline
\small{pop:individual:71} & \small{[79, 5, 0, 0, 1, 1, 2, 1, 0, 1, 0, 1, 1, 1, 3]}
& \small{16} & \small{16}\\ \hline
\small{pop:individual:42} & \small{[112, 35, 1, 0, 1, 0, 1, 0, 1, 0, 0, 1, 1, 0, 2]}
& \small{9} & \small{9}\\ \hline
\small{pop:individual:411} & \small{[113, 58, 0, 1, 1, 1, 4, 0, 0, 1, 0, 0, 1, 2, 2]}
& \small{16} & \small{15}\\ \hline
\small{pop:individual:59} & \small{[93, 23, 0, 0, 0, 1, 4, 0, 0, 0, 1, 0, 1, 2, 1]}
& \small{17} & \small{15}\\ \hline
\small{pop:individual:40} & \small{[137, 9, 1, 0, 1, 0, 3, 0, 1, 1, 2, 0, 1, 2, 1]}
& \small{15} & \small{15}\\ \hline
\small{pop:individual:85} & \small{[120, 42, 0, 1, 0, 1, 0, 0, 1, 1, 3, 0, 1, 1, 2]}
& \small{16} & \small{15}\\ \hline
\small{pop:individual:90} & \small{[130, 76, 0, 0, 1, 1, 1, 1, 1, 1, 1, 0, 1, 2, 3]}
& \small{15} & \small{15}\\ \hline
\small{pop:individual:73} & \small{[88, 67, 0, 0, 0, 0, 0, 1, 1, 1, 2, 1, 0, 1, 3]}
& \small{15} & \small{15}\\ \hline
\small{pop:individual:943} & \small{[116, 44, 0, 0, 0, 4, 1, 1, 1, 1, 0, 1, 0, 1, 1]}
& \small{15} & \small{15}\\ \hline
\small{pop:individual:353} & \small{[121, 20, 1, 1, 0, 0, 1, 0, 1, 1, 3, 0, 1, 0, 2]}
& \small{15} & \small{15}\\ \hline
\small{pop:individual:79} & \small{[68, 5, 1, 1, 0, 1, 1, 1, 1, 0, 2, 1, 1, 1, 1]}
& \small{14} & \small{14}\\ \hline
\noalign{\smallskip}\hline
\end{tabular}
\end{table}

Table \ref{tab:totalIndividuals_32} contains a sample of individuals 30/556
generated in the experiment. In this its unique identifier of the individual,
its chromosome, as well as the number of views, likes available to the
individual presents. This results which are useful to observe individuals have
been better evaluated by users.

\begin{table}
\small
\caption{Level of user participation.}
\label{tab:userParticipation_22}
\centering
\small
\begin{tabular}{p{4cm} p{4cm}}
\hline\noalign{\smallskip}
 User name & Participation   \\
\noalign{\smallskip}\hline\noalign{\smallskip}
\small{1157401414272355} & \small{2035} \\ \hline
\small{1001585659925992} & \small{1828} \\ \hline
\small{1244770712203948} & \small{1722} \\ \hline
\small{990632971026794} & \small{552} \\ \hline
\small{987920194610578} & \small{456} \\ \hline
\small{10207675081608741} & \small{300} \\ \hline
\small{966757780068615} & \small{267} \\ \hline
\small{1228561717171956} & \small{258} \\ \hline
\small{985416538190778} & \small{203} \\ \hline
\small{1123938694291507} & \small{202} \\ \hline
\small{10207552420841432} & \small{200} \\ \hline
\small{1124138344283213} & \small{169} \\ \hline
\small{220415891643957} & \small{161} \\ \hline
\small{534039336778842} & \small{112} \\ \hline
\small{10153940958696462} & \small{93} \\ \hline
\small{10205543461172072} & \small{87} \\ \hline
\small{1281817738500333} & \small{81} \\ \hline
\small{10153866481615259} & \small{74} \\ \hline
\small{943775989063530} & \small{73} \\ \hline
\small{10205664691039169} & \small{71} \\ \hline


\noalign{\smallskip}\hline
\end{tabular}
\end{table}

In figure \ref{fig:userP_3} shows a visual representation of user participation
of this experiment where the y-axis represents the level of participation and
the x axis represents the number of users who participated in this experiment.

\begin{figure*}
%\captionsetup{justification=centering,margin=1cm}
\centering
\fbox{\includegraphics[scale=0.75]{img/visual_representation_3.PNG}} %[width=0.7\textwidth]
\caption{Visual representation of user participation in EvoDrawing01.}
\label{fig:userP_3}
\end{figure*}


\begin{figure*}
%\captionsetup{justification=centering,margin=1cm}
\centering
\fbox{\includegraphics[scale=0.75]{img/weibull_3.PNG}} %[width=0.7\textwidth]
\caption{Weibull fit data representation.}
\label{fig:weibull_3}
\end{figure*}


\section{Comparison between experiments.}
In chart \ref{fig:comparison} shows a comparative graph of the results obtained
in the three phases of the study case, each of the lines on the graph represents
one of the versions of EvoDrawing in relation to the number of units of users
that were obtained are presented in different experiments. For instance the blue
line represents EvoDrawing01 experiment, the red line represents EvoDrawing02
and finally the green line represets EvoDrawing03.

\begin{figure*}
%\captionsetup{justification=centering,margin=1cm}
\centering
\fbox{\includegraphics[scale=0.75]{img/comparison.PNG}} %[width=0.7\textwidth]
\caption{Graphical representation of user participation in the different experiments.}
\label{fig:comparison}
\end{figure*}

\chapter{Conclusions and future work} \label{sec:5}

\section{Conclusions}

Using a user model in Web-based interactive evolutionary computation  overall with the different approaches such as fuzzy logic and gamification it demonstrated  in experiment EvoDrawing03 that the users increase their participation with respect to other versions (EvoDrawing01, EvoDrawin02).

In this sense the results have shown a phenomenon in the users which is competitiveness. This phenomenon occurs naturally because as human beings is our nature to be competitive regardless of the topic or activity that we assign [Reference]. This gave support to users return to evaluate more individuals within the experiment EvoDrawings03 and consequently the participation increase exponentially. 

In this research work also found that the way individuals was presented to be evaluated and how to evaluate them helped the user to take their evaluations so easy and quick. This means that users evaluated on average 40 or more individuals in an iteration, reducing the risk of demotivation assessment and therefore lose interest in participation. However, it was concluded that the biggest problem of interactive evolutionary computing systems remains on user fatigue.

The fatigue can be generated by many factors, such as how to evaluate individuals, the subject of the application, how to present the individuals, the objective within the application, expertise by users, and more. The resulting method of this research helps motivate users on the issue of participation in interactive evolutionary computing applications.


\section{Future work}



\prefacesection{Publications}
%\begin{singlespace}
\begin{enumerate}
\item \textit{User Modeling for Interactive Evolutionary Computation
Applications Using Fuzzy Logic. JC Romero, Mario Garc\'ia-Vald\'ez.
 Recent Advances on Hybrid Intelligent Systems. Springer Berlin Heidelberg. (2013)}
\item \textit{ EvoSpace-i: a framework for interactive evolutionary algorithms.
Mario Garc\'ia-Vald\'ez, 	Juan J. Merelo, 	Leonardo Trujillo, 	Francisco
Fernández-de-Vega Jos\'e C. Romero, 	Alejandra Mancilla. GECCO '13 Companion
Proceedings of the 15th annual conference  companion on Genetic and evolutionary
computation. Amsterdam, The Netherlands. (2013).}
\item \textit{Using a Graph Based Database to Support Collaborative Interactive
Evolutionary Systems.  JC Romero, Mario Garc\'ia-Vald\'ez. Recent Advances on
Hybrid Approaches for Designing Intelligent Systems. Springer International
Publishing Switzerland. Volume 547 of the series Studies in Computational
Intelligence pp 581-591. Switzerland. (2014).}
\item \textit{A Fitness Estimation Strategy for Web Based Interactive Evolutionary
Applications Considering User Preferences and Activities Using Fuzzy Logic. J.C.
Romero, Mario Garc\'ia-Vald\'ez. Design of Intelligent Systems Based on Fuzzy
Logic, Neural Networks and Nature-Inspired Optimization. Springer International
Publishing Switzerland. Volume 601 of the series Studies in Computational
Intelligence pp 507-516. (2015).}
\end{enumerate}
%\end{singlespace}

\appendix
\chapter{EvoDrawing deploy instructions}\label{apendixa}



\chapter{Fuzzy inference system IF-THEN rules for ED03 }\label{apendixb}

\begin{enumerate}
	\item \textit{If \textbf{preference} is low and 
		\textbf{experience} is low and \textbf{ranking} is low then \textbf{fuzzy-rate} is bad.}
	\item \textit{If \textbf{preference} is low and 
		\textbf{experience} is low and \textbf{ranking} is mid then \textbf{fuzzy-rate} is bad.}
	\item \textit{If \textbf{preference} is low and 
		\textbf{experience} is low and \textbf{ranking} is high then \textbf{fuzzy-rate} is bad.}
	\item \textit{If \textbf{preference} is low and 
		\textbf{experience} is mid and \textbf{ranking} is low then \textbf{fuzzy-rate} is bad.}
	\item \textit{If \textbf{preference} is low and 
		\textbf{experience} is mid and \textbf{ranking} is mid then \textbf{fuzzy-rate} is bad.}
	\item \textit{If \textbf{preference} is low and 
		\textbf{experience} is mid and \textbf{ranking} is high then \textbf{fuzzy-rate} is normal.}
	\item \textit{If \textbf{preference} is low and 
		\textbf{experience} is high and \textbf{ranking} is low then \textbf{fuzzy-rate} is normal.}
	\item \textit{If \textbf{preference} is low and 
		\textbf{experience} is high and \textbf{ranking} is mid then \textbf{fuzzy-rate} is normal.}
	\item \textit{If \textbf{preference} is low and 
		\textbf{experience} is high and \textbf{ranking} is high then \textbf{fuzzy-rate} is normal.}
	\item \textit{If \textbf{preference} is mid and 
		\textbf{experience} is low and \textbf{ranking} is low then \textbf{fuzzy-rate} is bad.}
	\item \textit{If \textbf{preference} is mid and 
		\textbf{experience} is low and \textbf{ranking} is mid then \textbf{fuzzy-rate} is normal.}
	\item \textit{If \textbf{preference} is mid and 
		\textbf{experience} is low and \textbf{ranking} is high then \textbf{fuzzy-rate} is normal.}
	\item \textit{If \textbf{preference} is mid and 
		\textbf{experience} is mid and \textbf{ranking} is low then \textbf{fuzzy-rate} is normal.}
	\item \textit{If \textbf{preference} is mid and 
		\textbf{experience} is mid and \textbf{ranking} is mid then \textbf{fuzzy-rate} is normal.}
	\item \textit{If \textbf{preference} is mid and 
		\textbf{experience} is mid and \textbf{ranking} is high then \textbf{fuzzy-rate} is normal.}
	\item \textit{If \textbf{preference} is mid and 
		\textbf{experience} is high and \textbf{ranking} is low then \textbf{fuzzy-rate} is normal.}
	\item \textit{If \textbf{preference} is mid and 
		\textbf{experience} is high and \textbf{ranking} is mid then \textbf{fuzzy-rate} is normal.}
	\item \textit{If \textbf{preference} is mid and 
		\textbf{experience} is high and \textbf{ranking} is high then \textbf{fuzzy-rate} is normal.}
	\item \textit{If \textbf{preference} is high and 
		\textbf{experience} is low and \textbf{ranking} is low then \textbf{fuzzy-rate} is normal.}\
	\item \textit{If \textbf{preference} is high and 
		\textbf{experience} is low and \textbf{ranking} is mid then \textbf{fuzzy-rate} is normal.}\
	\item \textit{If \textbf{preference} is high and 
		\textbf{experience} is low and \textbf{ranking} is high then \textbf{fuzzy-rate} is good.}\
	\item \textit{If \textbf{preference} is high and 
		\textbf{experience} is mid and \textbf{ranking} is low then \textbf{fuzzy-rate} is normal.}\
	\item \textit{If \textbf{preference} is high and 
		\textbf{experience} is mid and \textbf{ranking} is mid then \textbf{fuzzy-rate} is normal.}\
	\item \textit{If \textbf{preference} is high and 
		\textbf{experience} is mid and \textbf{ranking} is high then \textbf{fuzzy-rate} is good.}\
	\item \textit{If \textbf{preference} is high and 
		\textbf{experience} is high and \textbf{ranking} is low then \textbf{fuzzy-rate} is good.}\
	\item \textit{If \textbf{preference} is high and 
		\textbf{experience} is high and \textbf{ranking} is mid then \textbf{fuzzy-rate} is good.}\
	\item \textit{If \textbf{preference} is high and 
		\textbf{experience} is high and \textbf{ranking} is high then \textbf{fuzzy-rate} is good.}\
	
\end{enumerate} 























%% Cap'itulos incluidos despues del comando \appendix aparecen como ap'endices
%% de la tesis.
%\include{apendiceB}
%\include{apendiceC}

%% Incluir la bibliograf'ia. Mirar el archivo "biblio.bib" para m'as detales
%% y un ejemplo.
\bibliographystyle{ieeetr}
\bibliography{biblio}

\end{document}
%\include{appendix}
