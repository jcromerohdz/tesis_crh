\chapter{State of the art} \label{}


Some 1986's Dawkins's research was the pioneer of a significant addition to the
1990s IEC algorithms research works\cite{}[Dawkins 1986].

There is two key research approach about his field:

Creative Approach: The Artificial Life (AL) was the base of creative approach.
AL uses complex algorithms for biological life models emulation. To perform this
task, it is needed to include some of the different techniques starting from
right image treatment. Good graphic creation as well as a great music and
quality sounds, [Sims 1991b], [Sims 1991c], [Sims 1994], [Dawkins 1986], [Disz
1997], [Unemi 2000]and [Unemi 2003].
 
Humanized technology approach: The concept of humanized technology approach
comes from the approach that is focused on the IEC algorithms interface, this is
the research of interaction between humans and computer systems. The main goal
of this was to reduce the user's fatigue and to promote the inputs and outputs
of algorithms to improve the efficiency of them. IEC has made his own way in
practical fields such as engineering, education,etc., [Parmee 1993], [Ventrella
1994a], [Takagi 1996], [Poli 1997], [Parmee 1998] and [Takagi 1998].

Computer graphics (CG) The Biomorph of Dawkins was the first IEC research, from
this research comes to many motivated works mostly about the Selfish Gene, come
of these works are:  [Ochoa 1998], [Mccormack 1993], and [Smith 2003].

In Dawkins work a conventional recursive algorithm was used as a baseline
maintaining the main target of trees with an L-system Lindenmayer.

Some 1986's Dawkins's research was the pioneer of the significant addition to
the 1990s IEC algorithms research works[Dawkins 1986].

There is two key research approach about his field:

Creative Approach: The Artificial Life (AL) was the base of creative approach.
AL uses complex algorithms for biological life models emulation. In order to
perform this task it is needed to include some of the different techniques
starting from right image treatment , good graphic creation as well as a great
music and quality sounds, [Sims 1991b], [Sims 1991c], [Sims 1994], [Dawkins
1986], [Disz 1997], [Unemi 2000]and [Unemi 2003].
 
Humanized technology approach: The concept of humanized technology approach
comes from the approach that is focused on the IEC algorithms interface, this is
the research of interaction between humans and computer systems. The main goal
of this was to reduce the user's fatigue and to promote the inputs and outputs
of algorithms to improve the efficiency of them. IEC has made his own way in
practical fields such as engineering, education,etc., [Parmee 1993], [Ventrella
1994a], [Takagi 1996], [Poli 1997], [Parmee 1998] and [Takagi 1998].


Computer graphics (CG) The Biomorph of Dawkins was the first IEC research, from
this research comes to many motivated works mostly about the Selfish Gene, come
of these works are:  [Ochoa 1998], [Mccormack 1993], and [Smith 2003].

In Dawkins work, a conventional recursive algorithm was used as a baseline
maintaining the main target of trees with an L-system (Lindenmayer). This same
L-system was the base for another experiment to create 2-D CG forms insects from
a system called Blind Watchmaker who used L-system angles from L-system output
intuitively selected; the creation was called biomorphs. These creations reach
his target with the multiple selections of the users based on their preferences;
all these selections acted like a natural adaptation filter.

We can find plenty of applications and works for fractal generation [Sims 1991a]
and [Sims 1992], [Baluja 1993] and [Baluja 1994], [Lund 1995], or [Angeline
1996],[Raynal 1999] and [Lutton 2003], for rendering in tridimensional, [Todd
1991],[Broughton 1997], [Das 1994] and [Tam 2002], for generation of virtual
creatures, [Sims 1994], [Rowland 2000], or aerodynamic surface design (wings),
[NGuyen 1993], [NGuyen 1994] and [NGuyen 1997].

We can discover more than one additional way to use this research in the
artistic field with several applications of IEC who are used for cartoon face
construction and animations matters, like Mutator [Todd 1991], [Todd 1994]
and[Todd 1999] or [Bentley 1999a].

The genetic programming (GP) applications offers a category called Interactive
Genetic Programming (IGP) with many examples of successful application in
tridimensional artwork for artistic animations or construction using
mathematical equations as CAVE [Das 1994], [Papka 1996] and [Disz 1997], [Sims
1991], [Sims 1991a], [Sims 1992],[Sims 1993] and [Min 2004]. As this work
consequence, Panspermia orPrimordial Dance was created.

Imagen

Imagen

The artistic field is only the first step of a great IEC implementation; it is
important to mention another relevant projects called Galapagos, [Sims 1997],
and SBART, [Unemi 2000]. The IEC application Galapagos Project is the exhibit in
Tokio Multimedia Museum, (NTT Intercommunication Center) and this project
originates engaging images to all visitors based on L-systems.

Imagen

There are created after one selection, to get a good solution through multiple
repetitions. This action is performed with Genetic Programming (GP), after the
calculation of each pixel value using trees of equations combining logarithm,
maximum, and minimum, sine, root, cosine, exponential arithmetic operators.
AnimationLab is found as an outstanding work who offer figures that can run or
walk working with the user to receive more opportunities to be picked. A
particular characteristic of all of the figures is that the figures extremities
Mentioning open source works, we can find SBART as an IGP [Unemi 2000] tool to
create graphics. SBART allow to users to evaluate 20 two-dimensional images,
subsequently twenty new image has direction and angles.

Imagen

There are many examples for this field application as [McKenna 1990], [Ventrella
1994a], [Ventrella 1994b], or [Ventrella 1995], [Lim 1999] and [Lim 2000].  One
of the Interactive Evolutionary Programming (IEP)  artistic application was
created by [Angeline 1996], as a fractal generation where the system allows the
evolution of animations for the ones who were selected from the user, the
application initially show only 10 animations to rate.

Music and sound

It is important to know how IEC was implemented in music generation, with
several applications in this field. We will start mentioning the pioneer
application GENJAM, [Biles 1994], [Biles 1996] or [Biles 1999] and [Biles 2000].
Some other attractive works are Sonomorph, [Nelson 1993] and[Nelson 1995], or
SBEAT, [Unemi 2003], [Horowitz 1994], [Onisawa 2000], [Tokui 2000] and [Fels
2002]. It is possible to hear a part of the music songs of these previously
mentioned works broadcasted in the radio station WDYN. (100.1, New York, USA,
WEBPage:http://www.wdyn.net/).

The IEC algorithms are the base for the functionality of the music generation
systems, a visual representation of this is given in the below figure:

Imagen

\begin{sidewaystable}[]
  \caption{Comparison of context-aware recommender systems.}
    \label{tab:stateoftheart}
  \bigskip
    \centering\small\setlength\tabcolsep{2pt}
        \hspace*{-1cm}\begin{tabular}{p{3.5cm} p{6cm} p{4cm} p{3cm} p{3cm} }%{l l l l l}
           \toprule
             \textbf{Application} &\textbf{Contextual Factor} &\textbf{Domain} &\textbf{Paradigm} &\textbf{Device}  \\ \hline

           \midrule
             \textbf{CoMoLE} & \textbf{Time, available time, place, device, level of knowledge, learning style.} & \textbf{E-learning} & \textbf{Pre-filtering} & \textbf{Mobiles, PC, laptop.}   \\ \hline 

             \textbf{Moma-System} & \textbf{Location, time.} & \textbf{E-commerce} & \textbf{Post-filtering} & \textbf{PC, laptop.}  \\ \hline

             \textbf{UbiquITO} & \textbf{Season, time, temperature.} & \textbf{Tourism} & \textbf{Post-filtering} & \textbf{Mobiles} \\ \hline

             \textbf{ReRex} & \textbf{Distance of the point of interest,  temperature, weather, season, weekend, companion, travel goal, transport.} & \textbf{Tourism} & \textbf{Model-based} & \textbf{Mobiles} \\ \hline

             \textbf{LifeTrack} & \textbf{Location, time, day of the week, traifc noise(level), temperature, weather.} & \textbf{Music} & \textbf{ Post-filtering} & \textbf{PC, Mobiles.} \\ \hline

             \textbf{CARS} & \textbf{Location and season.} & \textbf{Restaurants} & \textbf{Post-filtering} & \textbf{PC, laptop.} \\ \hline

             \textbf{InCarMusic} & \textbf{Driving style, road type, landscape, sleepiness, traffic conditions, mood weather and natural phenomena.} & \textbf{Music} & \textbf{Model-based} & \textbf{Mobiles} \\ \hline

            \textbf{REJA} & \textbf{Location.} & \textbf{Restaurants} & \textbf{Pre-filtering and Post-filtering} & \textbf{PC, laptop, mobiles.} \\ \hline

            \textbf{CiberGuide} & \textbf{Location, time, weather.} & \textbf{Tourism} & \textbf{Post-filtering} & \textbf{Mobiles} \\ \hline

            \textbf{PECITAS} & \textbf{Location, routes.} & \textbf{Transport} & \textbf{Post-filtering} & \textbf{Mobiles} \\ \hline

            \textbf{LARS} & \textbf{Tourists’ location and time.} & \textbf{Tourist packages} & \textbf{Post-filtering} & \textbf{Mobiles} \\ \hline

            \textbf{I'm feeling LoCo} & \textbf{Location, transportation.} & \textbf{Tourism} & \textbf{Model-based} & \textbf{Mobiles} \\ \hline

            \textbf{MOPSI} & \textbf{Location} & \textbf{Tourism and transport} & \textbf{Post-filtering} & \textbf{Mobiles} \\ \hline

           \bottomrule
        \end{tabular}\hspace*{-1cm}
\end{sidewaystable}








