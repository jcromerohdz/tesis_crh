\chapter{EvoDrawing deploy instructions}\label{apendixa}

This section presents the instructions for deploying the EvoDrawings application
in the heroku cloud service.

\begin{enumerate}


\item \textit{First you need to clone the evodrawings
applications from github repository like this: In command promp put this:
git clone \url{https://github.com/jcromerohdz/evoDrawing_gaming.git/}. }


\item \textit{
Generate a repository on github service, if you don’t know how to create a
repository on github, here is a reference to help you create one \cite{}}

\item\textit{Generate an account on Heroku site.}

\item \textit{Go to applications and create one, this application can contain
any name you want in our particular case is evodrawings, after that you proceed
to link the heroku application with github service. Heroku service it’s gonna
ask you to provide permissions in order to connect to the github repository.}

\item \textit{Then find the manual deploy option on the application settings on
heroku.}

\item \textit{Now you have to enable two add-ons services on heroku.
First one is the graphene service, you can choose the free one
The secound is the redis to go service, also you can choose the free one.}

\item \textit{To do the previous you have to do the following, first go to your heroku
settings and find the add-on option, the finds the service you want and select
the plan for both services.}





\end{enumerate}
